% Options for packages loaded elsewhere
% Options for packages loaded elsewhere
\PassOptionsToPackage{unicode}{hyperref}
\PassOptionsToPackage{hyphens}{url}
\PassOptionsToPackage{dvipsnames,svgnames,x11names}{xcolor}
%
\documentclass[
  letterpaper,
  DIV=11,
  numbers=noendperiod]{scrartcl}
\usepackage{xcolor}
\usepackage{amsmath,amssymb}
\setcounter{secnumdepth}{-\maxdimen} % remove section numbering
\usepackage{iftex}
\ifPDFTeX
  \usepackage[T1]{fontenc}
  \usepackage[utf8]{inputenc}
  \usepackage{textcomp} % provide euro and other symbols
\else % if luatex or xetex
  \usepackage{unicode-math} % this also loads fontspec
  \defaultfontfeatures{Scale=MatchLowercase}
  \defaultfontfeatures[\rmfamily]{Ligatures=TeX,Scale=1}
\fi
\usepackage{lmodern}
\ifPDFTeX\else
  % xetex/luatex font selection
\fi
% Use upquote if available, for straight quotes in verbatim environments
\IfFileExists{upquote.sty}{\usepackage{upquote}}{}
\IfFileExists{microtype.sty}{% use microtype if available
  \usepackage[]{microtype}
  \UseMicrotypeSet[protrusion]{basicmath} % disable protrusion for tt fonts
}{}
\makeatletter
\@ifundefined{KOMAClassName}{% if non-KOMA class
  \IfFileExists{parskip.sty}{%
    \usepackage{parskip}
  }{% else
    \setlength{\parindent}{0pt}
    \setlength{\parskip}{6pt plus 2pt minus 1pt}}
}{% if KOMA class
  \KOMAoptions{parskip=half}}
\makeatother
% Make \paragraph and \subparagraph free-standing
\makeatletter
\ifx\paragraph\undefined\else
  \let\oldparagraph\paragraph
  \renewcommand{\paragraph}{
    \@ifstar
      \xxxParagraphStar
      \xxxParagraphNoStar
  }
  \newcommand{\xxxParagraphStar}[1]{\oldparagraph*{#1}\mbox{}}
  \newcommand{\xxxParagraphNoStar}[1]{\oldparagraph{#1}\mbox{}}
\fi
\ifx\subparagraph\undefined\else
  \let\oldsubparagraph\subparagraph
  \renewcommand{\subparagraph}{
    \@ifstar
      \xxxSubParagraphStar
      \xxxSubParagraphNoStar
  }
  \newcommand{\xxxSubParagraphStar}[1]{\oldsubparagraph*{#1}\mbox{}}
  \newcommand{\xxxSubParagraphNoStar}[1]{\oldsubparagraph{#1}\mbox{}}
\fi
\makeatother

\usepackage{color}
\usepackage{fancyvrb}
\newcommand{\VerbBar}{|}
\newcommand{\VERB}{\Verb[commandchars=\\\{\}]}
\DefineVerbatimEnvironment{Highlighting}{Verbatim}{commandchars=\\\{\}}
% Add ',fontsize=\small' for more characters per line
\usepackage{framed}
\definecolor{shadecolor}{RGB}{241,243,245}
\newenvironment{Shaded}{\begin{snugshade}}{\end{snugshade}}
\newcommand{\AlertTok}[1]{\textcolor[rgb]{0.68,0.00,0.00}{#1}}
\newcommand{\AnnotationTok}[1]{\textcolor[rgb]{0.37,0.37,0.37}{#1}}
\newcommand{\AttributeTok}[1]{\textcolor[rgb]{0.40,0.45,0.13}{#1}}
\newcommand{\BaseNTok}[1]{\textcolor[rgb]{0.68,0.00,0.00}{#1}}
\newcommand{\BuiltInTok}[1]{\textcolor[rgb]{0.00,0.23,0.31}{#1}}
\newcommand{\CharTok}[1]{\textcolor[rgb]{0.13,0.47,0.30}{#1}}
\newcommand{\CommentTok}[1]{\textcolor[rgb]{0.37,0.37,0.37}{#1}}
\newcommand{\CommentVarTok}[1]{\textcolor[rgb]{0.37,0.37,0.37}{\textit{#1}}}
\newcommand{\ConstantTok}[1]{\textcolor[rgb]{0.56,0.35,0.01}{#1}}
\newcommand{\ControlFlowTok}[1]{\textcolor[rgb]{0.00,0.23,0.31}{\textbf{#1}}}
\newcommand{\DataTypeTok}[1]{\textcolor[rgb]{0.68,0.00,0.00}{#1}}
\newcommand{\DecValTok}[1]{\textcolor[rgb]{0.68,0.00,0.00}{#1}}
\newcommand{\DocumentationTok}[1]{\textcolor[rgb]{0.37,0.37,0.37}{\textit{#1}}}
\newcommand{\ErrorTok}[1]{\textcolor[rgb]{0.68,0.00,0.00}{#1}}
\newcommand{\ExtensionTok}[1]{\textcolor[rgb]{0.00,0.23,0.31}{#1}}
\newcommand{\FloatTok}[1]{\textcolor[rgb]{0.68,0.00,0.00}{#1}}
\newcommand{\FunctionTok}[1]{\textcolor[rgb]{0.28,0.35,0.67}{#1}}
\newcommand{\ImportTok}[1]{\textcolor[rgb]{0.00,0.46,0.62}{#1}}
\newcommand{\InformationTok}[1]{\textcolor[rgb]{0.37,0.37,0.37}{#1}}
\newcommand{\KeywordTok}[1]{\textcolor[rgb]{0.00,0.23,0.31}{\textbf{#1}}}
\newcommand{\NormalTok}[1]{\textcolor[rgb]{0.00,0.23,0.31}{#1}}
\newcommand{\OperatorTok}[1]{\textcolor[rgb]{0.37,0.37,0.37}{#1}}
\newcommand{\OtherTok}[1]{\textcolor[rgb]{0.00,0.23,0.31}{#1}}
\newcommand{\PreprocessorTok}[1]{\textcolor[rgb]{0.68,0.00,0.00}{#1}}
\newcommand{\RegionMarkerTok}[1]{\textcolor[rgb]{0.00,0.23,0.31}{#1}}
\newcommand{\SpecialCharTok}[1]{\textcolor[rgb]{0.37,0.37,0.37}{#1}}
\newcommand{\SpecialStringTok}[1]{\textcolor[rgb]{0.13,0.47,0.30}{#1}}
\newcommand{\StringTok}[1]{\textcolor[rgb]{0.13,0.47,0.30}{#1}}
\newcommand{\VariableTok}[1]{\textcolor[rgb]{0.07,0.07,0.07}{#1}}
\newcommand{\VerbatimStringTok}[1]{\textcolor[rgb]{0.13,0.47,0.30}{#1}}
\newcommand{\WarningTok}[1]{\textcolor[rgb]{0.37,0.37,0.37}{\textit{#1}}}

\usepackage{longtable,booktabs,array}
\usepackage{calc} % for calculating minipage widths
% Correct order of tables after \paragraph or \subparagraph
\usepackage{etoolbox}
\makeatletter
\patchcmd\longtable{\par}{\if@noskipsec\mbox{}\fi\par}{}{}
\makeatother
% Allow footnotes in longtable head/foot
\IfFileExists{footnotehyper.sty}{\usepackage{footnotehyper}}{\usepackage{footnote}}
\makesavenoteenv{longtable}
\usepackage{graphicx}
\makeatletter
\newsavebox\pandoc@box
\newcommand*\pandocbounded[1]{% scales image to fit in text height/width
  \sbox\pandoc@box{#1}%
  \Gscale@div\@tempa{\textheight}{\dimexpr\ht\pandoc@box+\dp\pandoc@box\relax}%
  \Gscale@div\@tempb{\linewidth}{\wd\pandoc@box}%
  \ifdim\@tempb\p@<\@tempa\p@\let\@tempa\@tempb\fi% select the smaller of both
  \ifdim\@tempa\p@<\p@\scalebox{\@tempa}{\usebox\pandoc@box}%
  \else\usebox{\pandoc@box}%
  \fi%
}
% Set default figure placement to htbp
\def\fps@figure{htbp}
\makeatother





\setlength{\emergencystretch}{3em} % prevent overfull lines

\providecommand{\tightlist}{%
  \setlength{\itemsep}{0pt}\setlength{\parskip}{0pt}}



 


\KOMAoption{captions}{tableheading}
\makeatletter
\@ifpackageloaded{caption}{}{\usepackage{caption}}
\AtBeginDocument{%
\ifdefined\contentsname
  \renewcommand*\contentsname{Table of contents}
\else
  \newcommand\contentsname{Table of contents}
\fi
\ifdefined\listfigurename
  \renewcommand*\listfigurename{List of Figures}
\else
  \newcommand\listfigurename{List of Figures}
\fi
\ifdefined\listtablename
  \renewcommand*\listtablename{List of Tables}
\else
  \newcommand\listtablename{List of Tables}
\fi
\ifdefined\figurename
  \renewcommand*\figurename{Figure}
\else
  \newcommand\figurename{Figure}
\fi
\ifdefined\tablename
  \renewcommand*\tablename{Table}
\else
  \newcommand\tablename{Table}
\fi
}
\@ifpackageloaded{float}{}{\usepackage{float}}
\floatstyle{ruled}
\@ifundefined{c@chapter}{\newfloat{codelisting}{h}{lop}}{\newfloat{codelisting}{h}{lop}[chapter]}
\floatname{codelisting}{Listing}
\newcommand*\listoflistings{\listof{codelisting}{List of Listings}}
\makeatother
\makeatletter
\makeatother
\makeatletter
\@ifpackageloaded{caption}{}{\usepackage{caption}}
\@ifpackageloaded{subcaption}{}{\usepackage{subcaption}}
\makeatother
\usepackage{bookmark}
\IfFileExists{xurl.sty}{\usepackage{xurl}}{} % add URL line breaks if available
\urlstyle{same}
\hypersetup{
  pdftitle={Your First Visualization},
  pdfauthor={Dr.~Sara Weston},
  colorlinks=true,
  linkcolor={blue},
  filecolor={Maroon},
  citecolor={Blue},
  urlcolor={Blue},
  pdfcreator={LaTeX via pandoc}}


\title{Your First Visualization}
\usepackage{etoolbox}
\makeatletter
\providecommand{\subtitle}[1]{% add subtitle to \maketitle
  \apptocmd{\@title}{\par {\large #1 \par}}{}{}
}
\makeatother
\subtitle{PSY 410: Data Science for Psychology}
\author{Dr.~Sara Weston}
\date{2026-04-01}
\begin{document}
\maketitle


\section{Why visualize?}\label{why-visualize}

\subsection{Anscombe's Quartet}\label{anscombes-quartet}

Four datasets with identical summary statistics:

\begin{longtable}[]{@{}lrrrrr@{}}
\toprule\noalign{}
set & mean\_x & mean\_y & sd\_x & sd\_y & cor \\
\midrule\noalign{}
\endhead
\bottomrule\noalign{}
\endlastfoot
1 & 9 & 7.5 & 3.32 & 2.03 & 0.82 \\
2 & 9 & 7.5 & 3.32 & 2.03 & 0.82 \\
3 & 9 & 7.5 & 3.32 & 2.03 & 0.82 \\
4 & 9 & 7.5 & 3.32 & 2.03 & 0.82 \\
\end{longtable}

\subsection{But look at the plots!}\label{but-look-at-the-plots}

\pandocbounded{\includegraphics[keepaspectratio]{02-first-visualization_files/figure-pdf/unnamed-chunk-3-1.pdf}}

\textbf{Always visualize your data before running statistics.}

\subsection{The datasaurus dozen}\label{the-datasaurus-dozen}

\pandocbounded{\includegraphics[keepaspectratio]{02-first-visualization_files/figure-pdf/unnamed-chunk-4-1.pdf}}

\section{Introduction to ggplot2}\label{introduction-to-ggplot2}

\subsection{What is ggplot2?}\label{what-is-ggplot2}

\begin{itemize}
\tightlist
\item
  Created by Hadley Wickham (2005)
\item
  Based on the \textbf{Grammar of Graphics} by Leland Wilkinson
\item
  Most popular R visualization package
\item
  Part of the tidyverse
\end{itemize}

. . .

The ``gg'' stands for ``Grammar of Graphics''

\subsection{The grammar of graphics}\label{the-grammar-of-graphics}

Every ggplot has three essential components:

\begin{enumerate}
\def\labelenumi{\arabic{enumi}.}
\tightlist
\item
  \textbf{Data} --- what you want to visualize
\item
  \textbf{Aesthetics (aes)} --- how variables map to visual properties
\item
  \textbf{Geoms} --- what geometric shapes represent the data
\end{enumerate}

. . .

\begin{Shaded}
\begin{Highlighting}[]
\CommentTok{\# The basic template}
\FunctionTok{ggplot}\NormalTok{(}\AttributeTok{data =} \SpecialCharTok{\textless{}}\NormalTok{DATA}\SpecialCharTok{\textgreater{}}\NormalTok{, }\AttributeTok{mapping =} \FunctionTok{aes}\NormalTok{(}\SpecialCharTok{\textless{}}\NormalTok{MAPPINGS}\SpecialCharTok{\textgreater{}}\NormalTok{)) }\SpecialCharTok{+}
  \ErrorTok{\textless{}}\NormalTok{GEOM\_FUNCTION}\SpecialCharTok{\textgreater{}}\NormalTok{()}
\end{Highlighting}
\end{Shaded}

\subsection{Our dataset: mpg}\label{our-dataset-mpg}

\begin{Shaded}
\begin{Highlighting}[]
\CommentTok{\# Fuel economy data for 234 cars}
\FunctionTok{glimpse}\NormalTok{(mpg)}
\end{Highlighting}
\end{Shaded}

\begin{verbatim}
Rows: 234
Columns: 11
$ manufacturer <chr> "audi", "audi", "audi", "audi", "audi", "audi", "audi", "~
$ model        <chr> "a4", "a4", "a4", "a4", "a4", "a4", "a4", "a4 quattro", "~
$ displ        <dbl> 1.8, 1.8, 2.0, 2.0, 2.8, 2.8, 3.1, 1.8, 1.8, 2.0, 2.0, 2.~
$ year         <int> 1999, 1999, 2008, 2008, 1999, 1999, 2008, 1999, 1999, 200~
$ cyl          <int> 4, 4, 4, 4, 6, 6, 6, 4, 4, 4, 4, 6, 6, 6, 6, 6, 6, 8, 8, ~
$ trans        <chr> "auto(l5)", "manual(m5)", "manual(m6)", "auto(av)", "auto~
$ drv          <chr> "f", "f", "f", "f", "f", "f", "f", "4", "4", "4", "4", "4~
$ cty          <int> 18, 21, 20, 21, 16, 18, 18, 18, 16, 20, 19, 15, 17, 17, 1~
$ hwy          <int> 29, 29, 31, 30, 26, 26, 27, 26, 25, 28, 27, 25, 25, 25, 2~
$ fl           <chr> "p", "p", "p", "p", "p", "p", "p", "p", "p", "p", "p", "p~
$ class        <chr> "compact", "compact", "compact", "compact", "compact", "c~
\end{verbatim}

\subsection{Your first plot}\label{your-first-plot}

\begin{Shaded}
\begin{Highlighting}[]
\CommentTok{\# Relationship between engine size and highway mpg}
\FunctionTok{ggplot}\NormalTok{(}\AttributeTok{data =}\NormalTok{ mpg, }\AttributeTok{mapping =} \FunctionTok{aes}\NormalTok{(}\AttributeTok{x =}\NormalTok{ displ, }\AttributeTok{y =}\NormalTok{ hwy)) }\SpecialCharTok{+}
  \FunctionTok{geom\_point}\NormalTok{()}
\end{Highlighting}
\end{Shaded}

\pandocbounded{\includegraphics[keepaspectratio]{02-first-visualization_files/figure-pdf/unnamed-chunk-7-1.pdf}}

\subsection{Breaking it down}\label{breaking-it-down}

\begin{Shaded}
\begin{Highlighting}[numbers=left,,]
\FunctionTok{ggplot}\NormalTok{(}\AttributeTok{data =}\NormalTok{ mpg, }\AttributeTok{mapping =} \FunctionTok{aes}\NormalTok{(}\AttributeTok{x =}\NormalTok{ displ, }\AttributeTok{y =}\NormalTok{ hwy)) }\SpecialCharTok{+}
  \CommentTok{\# ↑ Data and aesthetic mappings}
  \FunctionTok{geom\_point}\NormalTok{()}
  \CommentTok{\# ↑ Geometric object (points = scatterplot)}
\end{Highlighting}
\end{Shaded}

\begin{itemize}
\tightlist
\item
  \texttt{data\ =\ mpg} --- use the mpg dataset
\item
  \texttt{aes(x\ =\ displ,\ y\ =\ hwy)} --- map displacement to x,
  highway mpg to y
\item
  \texttt{geom\_point()} --- represent data as points
\end{itemize}

\subsection{A cleaner way to write it}\label{a-cleaner-way-to-write-it}

You can drop \texttt{data\ =} and \texttt{mapping\ =}:

\begin{Shaded}
\begin{Highlighting}[]
\CommentTok{\# These are equivalent}
\FunctionTok{ggplot}\NormalTok{(}\AttributeTok{data =}\NormalTok{ mpg, }\AttributeTok{mapping =} \FunctionTok{aes}\NormalTok{(}\AttributeTok{x =}\NormalTok{ displ, }\AttributeTok{y =}\NormalTok{ hwy)) }\SpecialCharTok{+} \FunctionTok{geom\_point}\NormalTok{()}

\FunctionTok{ggplot}\NormalTok{(mpg, }\FunctionTok{aes}\NormalTok{(}\AttributeTok{x =}\NormalTok{ displ, }\AttributeTok{y =}\NormalTok{ hwy)) }\SpecialCharTok{+} \FunctionTok{geom\_point}\NormalTok{()}

\FunctionTok{ggplot}\NormalTok{(mpg, }\FunctionTok{aes}\NormalTok{(displ, hwy)) }\SpecialCharTok{+} \FunctionTok{geom\_point}\NormalTok{()  }\CommentTok{\# x and y are first args}
\end{Highlighting}
\end{Shaded}

I'll use the middle version --- clear but not overly verbose.

\section{Aesthetic mappings}\label{aesthetic-mappings}

\subsection{What are aesthetics?}\label{what-are-aesthetics}

\textbf{Aesthetics} are visual properties of geoms:

\begin{itemize}
\tightlist
\item
  \texttt{x}, \texttt{y} --- position
\item
  \texttt{color} --- outline color
\item
  \texttt{fill} --- interior color
\item
  \texttt{size} --- how big
\item
  \texttt{shape} --- what shape
\item
  \texttt{alpha} --- transparency
\end{itemize}

\subsection{Mapping color to a
variable}\label{mapping-color-to-a-variable}

What if we want to see which points are which car class?

\begin{Shaded}
\begin{Highlighting}[]
\FunctionTok{ggplot}\NormalTok{(mpg, }\FunctionTok{aes}\NormalTok{(}\AttributeTok{x =}\NormalTok{ displ, }\AttributeTok{y =}\NormalTok{ hwy, }\AttributeTok{color =}\NormalTok{ class)) }\SpecialCharTok{+}
  \FunctionTok{geom\_point}\NormalTok{()}
\end{Highlighting}
\end{Shaded}

\pandocbounded{\includegraphics[keepaspectratio]{02-first-visualization_files/figure-pdf/unnamed-chunk-10-1.pdf}}

\subsection{Mapping size to a
variable}\label{mapping-size-to-a-variable}

\begin{Shaded}
\begin{Highlighting}[]
\CommentTok{\# Size by number of cylinders}
\FunctionTok{ggplot}\NormalTok{(mpg, }\FunctionTok{aes}\NormalTok{(}\AttributeTok{x =}\NormalTok{ displ, }\AttributeTok{y =}\NormalTok{ hwy, }\AttributeTok{size =}\NormalTok{ cyl)) }\SpecialCharTok{+}
  \FunctionTok{geom\_point}\NormalTok{()}
\end{Highlighting}
\end{Shaded}

\pandocbounded{\includegraphics[keepaspectratio]{02-first-visualization_files/figure-pdf/unnamed-chunk-11-1.pdf}}

\subsection{Mapping alpha
(transparency)}\label{mapping-alpha-transparency}

\begin{Shaded}
\begin{Highlighting}[]
\FunctionTok{ggplot}\NormalTok{(mpg, }\FunctionTok{aes}\NormalTok{(}\AttributeTok{x =}\NormalTok{ displ, }\AttributeTok{y =}\NormalTok{ hwy, }\AttributeTok{alpha =}\NormalTok{ cyl)) }\SpecialCharTok{+}
  \FunctionTok{geom\_point}\NormalTok{()}
\end{Highlighting}
\end{Shaded}

\pandocbounded{\includegraphics[keepaspectratio]{02-first-visualization_files/figure-pdf/unnamed-chunk-12-1.pdf}}

\subsection{Mapping shape}\label{mapping-shape}

\begin{Shaded}
\begin{Highlighting}[]
\CommentTok{\# Shape by drive type (front, rear, 4wd)}
\FunctionTok{ggplot}\NormalTok{(mpg, }\FunctionTok{aes}\NormalTok{(}\AttributeTok{x =}\NormalTok{ displ, }\AttributeTok{y =}\NormalTok{ hwy, }\AttributeTok{shape =}\NormalTok{ drv)) }\SpecialCharTok{+}
  \FunctionTok{geom\_point}\NormalTok{()}
\end{Highlighting}
\end{Shaded}

\pandocbounded{\includegraphics[keepaspectratio]{02-first-visualization_files/figure-pdf/unnamed-chunk-13-1.pdf}}

\subsection{Combining multiple
aesthetics}\label{combining-multiple-aesthetics}

\begin{Shaded}
\begin{Highlighting}[]
\FunctionTok{ggplot}\NormalTok{(mpg, }\FunctionTok{aes}\NormalTok{(}\AttributeTok{x =}\NormalTok{ displ, }\AttributeTok{y =}\NormalTok{ hwy, }\AttributeTok{color =}\NormalTok{ class, }\AttributeTok{shape =}\NormalTok{ drv)) }\SpecialCharTok{+}
  \FunctionTok{geom\_point}\NormalTok{(}\AttributeTok{size =} \DecValTok{3}\NormalTok{)}
\end{Highlighting}
\end{Shaded}

\pandocbounded{\includegraphics[keepaspectratio]{02-first-visualization_files/figure-pdf/unnamed-chunk-14-1.pdf}}

\subsection{Setting vs.~mapping}\label{setting-vs.-mapping}

\textbf{Mapping} --- aesthetic varies with data (inside \texttt{aes()})

\begin{Shaded}
\begin{Highlighting}[]
\FunctionTok{ggplot}\NormalTok{(mpg, }\FunctionTok{aes}\NormalTok{(}\AttributeTok{x =}\NormalTok{ displ, }\AttributeTok{y =}\NormalTok{ hwy, }\AttributeTok{color =}\NormalTok{ class)) }\SpecialCharTok{+}
  \FunctionTok{geom\_point}\NormalTok{()}
\end{Highlighting}
\end{Shaded}

\textbf{Setting} --- aesthetic is constant (outside \texttt{aes()})

\begin{Shaded}
\begin{Highlighting}[]
\FunctionTok{ggplot}\NormalTok{(mpg, }\FunctionTok{aes}\NormalTok{(}\AttributeTok{x =}\NormalTok{ displ, }\AttributeTok{y =}\NormalTok{ hwy)) }\SpecialCharTok{+}
  \FunctionTok{geom\_point}\NormalTok{(}\AttributeTok{color =} \StringTok{"blue"}\NormalTok{, }\AttributeTok{size =} \DecValTok{3}\NormalTok{)}
\end{Highlighting}
\end{Shaded}

\subsection{Setting aesthetics
manually}\label{setting-aesthetics-manually}

\begin{Shaded}
\begin{Highlighting}[]
\FunctionTok{ggplot}\NormalTok{(mpg, }\FunctionTok{aes}\NormalTok{(}\AttributeTok{x =}\NormalTok{ displ, }\AttributeTok{y =}\NormalTok{ hwy)) }\SpecialCharTok{+}
  \FunctionTok{geom\_point}\NormalTok{(}\AttributeTok{color =} \StringTok{"steelblue"}\NormalTok{, }\AttributeTok{size =} \DecValTok{3}\NormalTok{, }\AttributeTok{alpha =} \FloatTok{0.6}\NormalTok{)}
\end{Highlighting}
\end{Shaded}

\pandocbounded{\includegraphics[keepaspectratio]{02-first-visualization_files/figure-pdf/unnamed-chunk-17-1.pdf}}

\subsection{Common mistake!}\label{common-mistake}

\begin{Shaded}
\begin{Highlighting}[]
\CommentTok{\# What happens if you put a constant inside aes()?}
\FunctionTok{ggplot}\NormalTok{(mpg, }\FunctionTok{aes}\NormalTok{(}\AttributeTok{x =}\NormalTok{ displ, }\AttributeTok{y =}\NormalTok{ hwy, }\AttributeTok{color =} \StringTok{"blue"}\NormalTok{)) }\SpecialCharTok{+}
  \FunctionTok{geom\_point}\NormalTok{()}
\end{Highlighting}
\end{Shaded}

\pandocbounded{\includegraphics[keepaspectratio]{02-first-visualization_files/figure-pdf/unnamed-chunk-18-1.pdf}}

ggplot thinks ``blue'' is a category name!

\section{Geoms}\label{geoms}

\subsection{Common geoms}\label{common-geoms}

\begin{longtable}[]{@{}ll@{}}
\toprule\noalign{}
Geom & What it makes \\
\midrule\noalign{}
\endhead
\bottomrule\noalign{}
\endlastfoot
\texttt{geom\_point()} & Scatterplot \\
\texttt{geom\_line()} & Line graph \\
\texttt{geom\_bar()} & Bar chart \\
\texttt{geom\_histogram()} & Histogram \\
\texttt{geom\_boxplot()} & Box plot \\
\texttt{geom\_smooth()} & Smoothed line \\
\end{longtable}

\subsection{geom\_smooth()}\label{geom_smooth}

Add a trend line to your scatterplot:

\begin{Shaded}
\begin{Highlighting}[]
\FunctionTok{ggplot}\NormalTok{(mpg, }\FunctionTok{aes}\NormalTok{(}\AttributeTok{x =}\NormalTok{ displ, }\AttributeTok{y =}\NormalTok{ hwy)) }\SpecialCharTok{+}
  \FunctionTok{geom\_point}\NormalTok{() }\SpecialCharTok{+}
  \FunctionTok{geom\_smooth}\NormalTok{()  }\CommentTok{\# Adds a smoothed trend line}
\end{Highlighting}
\end{Shaded}

\pandocbounded{\includegraphics[keepaspectratio]{02-first-visualization_files/figure-pdf/unnamed-chunk-19-1.pdf}}

\subsection{Linear trend line}\label{linear-trend-line}

\begin{Shaded}
\begin{Highlighting}[]
\FunctionTok{ggplot}\NormalTok{(mpg, }\FunctionTok{aes}\NormalTok{(}\AttributeTok{x =}\NormalTok{ displ, }\AttributeTok{y =}\NormalTok{ hwy)) }\SpecialCharTok{+}
  \FunctionTok{geom\_point}\NormalTok{() }\SpecialCharTok{+}
  \FunctionTok{geom\_smooth}\NormalTok{(}\AttributeTok{method =} \StringTok{"lm"}\NormalTok{)  }\CommentTok{\# lm = linear model}
\end{Highlighting}
\end{Shaded}

\pandocbounded{\includegraphics[keepaspectratio]{02-first-visualization_files/figure-pdf/unnamed-chunk-20-1.pdf}}

\subsection{Layering geoms}\label{layering-geoms}

Each \texttt{+} adds a layer:

\begin{Shaded}
\begin{Highlighting}[]
\FunctionTok{ggplot}\NormalTok{(mpg, }\FunctionTok{aes}\NormalTok{(}\AttributeTok{x =}\NormalTok{ displ, }\AttributeTok{y =}\NormalTok{ hwy, }\AttributeTok{color =}\NormalTok{ drv)) }\SpecialCharTok{+}
  \FunctionTok{geom\_point}\NormalTok{() }\SpecialCharTok{+}
  \FunctionTok{geom\_smooth}\NormalTok{(}\AttributeTok{method =} \StringTok{"lm"}\NormalTok{, }\AttributeTok{se =} \ConstantTok{FALSE}\NormalTok{)  }\CommentTok{\# se = FALSE removes confidence band}
\end{Highlighting}
\end{Shaded}

\pandocbounded{\includegraphics[keepaspectratio]{02-first-visualization_files/figure-pdf/unnamed-chunk-21-1.pdf}}

\subsection{geom\_bar() --- categorical
data}\label{geom_bar-categorical-data}

\begin{Shaded}
\begin{Highlighting}[]
\CommentTok{\# Counts of each car class}
\FunctionTok{ggplot}\NormalTok{(mpg, }\FunctionTok{aes}\NormalTok{(}\AttributeTok{x =}\NormalTok{ class)) }\SpecialCharTok{+}
  \FunctionTok{geom\_bar}\NormalTok{()}
\end{Highlighting}
\end{Shaded}

\pandocbounded{\includegraphics[keepaspectratio]{02-first-visualization_files/figure-pdf/unnamed-chunk-22-1.pdf}}

\subsection{geom\_histogram() ---
distributions}\label{geom_histogram-distributions}

\begin{Shaded}
\begin{Highlighting}[]
\CommentTok{\# Distribution of highway mpg}
\FunctionTok{ggplot}\NormalTok{(mpg, }\FunctionTok{aes}\NormalTok{(}\AttributeTok{x =}\NormalTok{ hwy)) }\SpecialCharTok{+}
  \FunctionTok{geom\_histogram}\NormalTok{(}\AttributeTok{binwidth =} \DecValTok{2}\NormalTok{, }\AttributeTok{fill =} \StringTok{"steelblue"}\NormalTok{, }\AttributeTok{color =} \StringTok{"white"}\NormalTok{)}
\end{Highlighting}
\end{Shaded}

\pandocbounded{\includegraphics[keepaspectratio]{02-first-visualization_files/figure-pdf/unnamed-chunk-23-1.pdf}}

\subsection{geom\_boxplot() --- comparing
groups}\label{geom_boxplot-comparing-groups}

\begin{Shaded}
\begin{Highlighting}[]
\CommentTok{\# Highway mpg by car class}
\FunctionTok{ggplot}\NormalTok{(mpg, }\FunctionTok{aes}\NormalTok{(}\AttributeTok{x =}\NormalTok{ class, }\AttributeTok{y =}\NormalTok{ hwy)) }\SpecialCharTok{+}
  \FunctionTok{geom\_boxplot}\NormalTok{()}
\end{Highlighting}
\end{Shaded}

\pandocbounded{\includegraphics[keepaspectratio]{02-first-visualization_files/figure-pdf/unnamed-chunk-24-1.pdf}}

\section{Facets}\label{facets}

\subsection{What are facets?}\label{what-are-facets}

\textbf{Facets} split your plot into small multiples based on a
variable.

\begin{Shaded}
\begin{Highlighting}[]
\FunctionTok{ggplot}\NormalTok{(mpg, }\FunctionTok{aes}\NormalTok{(}\AttributeTok{x =}\NormalTok{ displ, }\AttributeTok{y =}\NormalTok{ hwy)) }\SpecialCharTok{+}
  \FunctionTok{geom\_point}\NormalTok{() }\SpecialCharTok{+}
  \FunctionTok{facet\_wrap}\NormalTok{(}\SpecialCharTok{\textasciitilde{}}\NormalTok{class)}
\end{Highlighting}
\end{Shaded}

\pandocbounded{\includegraphics[keepaspectratio]{02-first-visualization_files/figure-pdf/unnamed-chunk-25-1.pdf}}

\subsection{facet\_wrap()}\label{facet_wrap}

Creates a ribbon of panels:

\begin{Shaded}
\begin{Highlighting}[]
\CommentTok{\# Syntax: facet\_wrap(\textasciitilde{}variable)}
\FunctionTok{ggplot}\NormalTok{(mpg, }\FunctionTok{aes}\NormalTok{(}\AttributeTok{x =}\NormalTok{ displ, }\AttributeTok{y =}\NormalTok{ hwy)) }\SpecialCharTok{+}
  \FunctionTok{geom\_point}\NormalTok{() }\SpecialCharTok{+}
  \FunctionTok{facet\_wrap}\NormalTok{(}\SpecialCharTok{\textasciitilde{}}\NormalTok{class)}
\end{Highlighting}
\end{Shaded}

\begin{itemize}
\tightlist
\item
  Use \texttt{ncol} or \texttt{nrow} to control layout
\item
  \texttt{scales\ =\ "free"} allows different axis ranges
\end{itemize}

\subsection{facet\_grid()}\label{facet_grid}

Creates a grid of panels with two variables:

\begin{Shaded}
\begin{Highlighting}[]
\FunctionTok{ggplot}\NormalTok{(mpg, }\FunctionTok{aes}\NormalTok{(}\AttributeTok{x =}\NormalTok{ displ, }\AttributeTok{y =}\NormalTok{ hwy)) }\SpecialCharTok{+}
  \FunctionTok{geom\_point}\NormalTok{() }\SpecialCharTok{+}
  \FunctionTok{facet\_grid}\NormalTok{(drv }\SpecialCharTok{\textasciitilde{}}\NormalTok{ cyl)}
\end{Highlighting}
\end{Shaded}

\pandocbounded{\includegraphics[keepaspectratio]{02-first-visualization_files/figure-pdf/unnamed-chunk-27-1.pdf}}

\subsection{When to use facets}\label{when-to-use-facets}

Facets are great when:

\begin{itemize}
\tightlist
\item
  You have too many colors/shapes to distinguish
\item
  You want to compare patterns across groups
\item
  You want each group to ``stand alone''
\end{itemize}

\section{Making it look good}\label{making-it-look-good}

\subsection{Adding labels}\label{adding-labels}

\begin{Shaded}
\begin{Highlighting}[]
\FunctionTok{ggplot}\NormalTok{(mpg, }\FunctionTok{aes}\NormalTok{(}\AttributeTok{x =}\NormalTok{ displ, }\AttributeTok{y =}\NormalTok{ hwy)) }\SpecialCharTok{+}
  \FunctionTok{geom\_point}\NormalTok{(}\FunctionTok{aes}\NormalTok{(}\AttributeTok{color =}\NormalTok{ class)) }\SpecialCharTok{+}
  \FunctionTok{labs}\NormalTok{(}
    \AttributeTok{title =} \StringTok{"Fuel Efficiency vs Engine Size"}\NormalTok{,}
    \AttributeTok{subtitle =} \StringTok{"Larger engines tend to be less efficient"}\NormalTok{,}
    \AttributeTok{x =} \StringTok{"Engine displacement (liters)"}\NormalTok{,}
    \AttributeTok{y =} \StringTok{"Highway fuel efficiency (mpg)"}\NormalTok{,}
    \AttributeTok{color =} \StringTok{"Vehicle class"}
\NormalTok{  )}
\end{Highlighting}
\end{Shaded}

\pandocbounded{\includegraphics[keepaspectratio]{02-first-visualization_files/figure-pdf/unnamed-chunk-28-1.pdf}}

\subsection{Themes}\label{themes}

Themes control the overall look:

\begin{Shaded}
\begin{Highlighting}[]
\FunctionTok{ggplot}\NormalTok{(mpg, }\FunctionTok{aes}\NormalTok{(}\AttributeTok{x =}\NormalTok{ displ, }\AttributeTok{y =}\NormalTok{ hwy)) }\SpecialCharTok{+}
  \FunctionTok{geom\_point}\NormalTok{(}\FunctionTok{aes}\NormalTok{(}\AttributeTok{color =}\NormalTok{ class)) }\SpecialCharTok{+}
  \FunctionTok{theme\_minimal}\NormalTok{()  }\CommentTok{\# Clean, minimal theme}
\end{Highlighting}
\end{Shaded}

\pandocbounded{\includegraphics[keepaspectratio]{02-first-visualization_files/figure-pdf/unnamed-chunk-29-1.pdf}}

\subsection{Built-in themes}\label{built-in-themes}

\pandocbounded{\includegraphics[keepaspectratio]{02-first-visualization_files/figure-pdf/unnamed-chunk-30-1.pdf}}

\subsection{Saving plots}\label{saving-plots}

Use \texttt{ggsave()} to save your plot:

\begin{Shaded}
\begin{Highlighting}[]
\CommentTok{\# Create the plot}
\NormalTok{my\_plot }\OtherTok{\textless{}{-}} \FunctionTok{ggplot}\NormalTok{(mpg, }\FunctionTok{aes}\NormalTok{(}\AttributeTok{x =}\NormalTok{ displ, }\AttributeTok{y =}\NormalTok{ hwy, }\AttributeTok{color =}\NormalTok{ class)) }\SpecialCharTok{+}
  \FunctionTok{geom\_point}\NormalTok{() }\SpecialCharTok{+}
  \FunctionTok{theme\_minimal}\NormalTok{()}

\CommentTok{\# Save it}
\FunctionTok{ggsave}\NormalTok{(}\StringTok{"my\_plot.png"}\NormalTok{, my\_plot, }\AttributeTok{width =} \DecValTok{8}\NormalTok{, }\AttributeTok{height =} \DecValTok{6}\NormalTok{)}
\end{Highlighting}
\end{Shaded}

\section{Putting it together}\label{putting-it-together}

\subsection{A complete example}\label{a-complete-example}

\begin{Shaded}
\begin{Highlighting}[]
\FunctionTok{ggplot}\NormalTok{(mpg, }\FunctionTok{aes}\NormalTok{(}\AttributeTok{x =}\NormalTok{ displ, }\AttributeTok{y =}\NormalTok{ hwy)) }\SpecialCharTok{+}
  \FunctionTok{geom\_point}\NormalTok{(}\FunctionTok{aes}\NormalTok{(}\AttributeTok{color =}\NormalTok{ class), }\AttributeTok{size =} \DecValTok{2}\NormalTok{, }\AttributeTok{alpha =} \FloatTok{0.7}\NormalTok{) }\SpecialCharTok{+}
  \FunctionTok{geom\_smooth}\NormalTok{(}\AttributeTok{method =} \StringTok{"lm"}\NormalTok{, }\AttributeTok{color =} \StringTok{"black"}\NormalTok{, }\AttributeTok{se =} \ConstantTok{TRUE}\NormalTok{) }\SpecialCharTok{+}
  \FunctionTok{labs}\NormalTok{(}
    \AttributeTok{title =} \StringTok{"Fuel Efficiency Decreases with Engine Size"}\NormalTok{,}
    \AttributeTok{subtitle =} \StringTok{"Data from 234 vehicles (1999{-}2008)"}\NormalTok{,}
    \AttributeTok{x =} \StringTok{"Engine displacement (liters)"}\NormalTok{,}
    \AttributeTok{y =} \StringTok{"Highway MPG"}\NormalTok{,}
    \AttributeTok{color =} \StringTok{"Vehicle class"}\NormalTok{,}
    \AttributeTok{caption =} \StringTok{"Source: EPA fuel economy data"}
\NormalTok{  ) }\SpecialCharTok{+}
  \FunctionTok{theme\_minimal}\NormalTok{(}\AttributeTok{base\_size =} \DecValTok{14}\NormalTok{)}
\end{Highlighting}
\end{Shaded}

\pandocbounded{\includegraphics[keepaspectratio]{02-first-visualization_files/figure-pdf/unnamed-chunk-32-1.pdf}}

\subsection{Your turn!}\label{your-turn}

Using the \texttt{mpg} dataset, create a scatterplot that:

\begin{enumerate}
\def\labelenumi{\arabic{enumi}.}
\tightlist
\item
  Shows city mpg (\texttt{cty}) vs highway mpg (\texttt{hwy})
\item
  Colors points by fuel type (\texttt{fl})
\item
  Adds a linear trend line
\item
  Has appropriate labels
\item
  Uses \texttt{theme\_bw()}
\end{enumerate}

\section{Wrapping up}\label{wrapping-up}

\subsection{The ggplot2 template}\label{the-ggplot2-template}

\begin{Shaded}
\begin{Highlighting}[]
\FunctionTok{ggplot}\NormalTok{(}\SpecialCharTok{\textless{}}\NormalTok{DATA}\SpecialCharTok{\textgreater{}}\NormalTok{, }\FunctionTok{aes}\NormalTok{(}\SpecialCharTok{\textless{}}\NormalTok{MAPPINGS}\SpecialCharTok{\textgreater{}}\NormalTok{)) }\SpecialCharTok{+}
  \ErrorTok{\textless{}}\NormalTok{GEOM\_FUNCTION}\SpecialCharTok{\textgreater{}}\NormalTok{() }\SpecialCharTok{+}
  \ErrorTok{\textless{}}\NormalTok{FACET\_FUNCTION}\SpecialCharTok{\textgreater{}}\NormalTok{() }\SpecialCharTok{+}
  \FunctionTok{labs}\NormalTok{(}\SpecialCharTok{\textless{}}\NormalTok{LABELS}\SpecialCharTok{\textgreater{}}\NormalTok{) }\SpecialCharTok{+}
\NormalTok{  theme\_}\SpecialCharTok{\textless{}}\NormalTok{THEME}\SpecialCharTok{\textgreater{}}\NormalTok{()}
\end{Highlighting}
\end{Shaded}

You can build almost any visualization with this template!

\subsection{Before next class}\label{before-next-class}

📖 \textbf{Read:}

\begin{itemize}
\tightlist
\item
  \href{https://r4ds.hadley.nz/data-transform}{R4DS Ch 3: Data
  transformation} (sections 3.1--3.4)
\end{itemize}

✅ \textbf{Practice:}

\begin{itemize}
\tightlist
\item
  Create 3 different plots with \texttt{mpg} or \texttt{diamonds}
\item
  Try different geoms, aesthetics, and facets
\item
  Save your favorite plot with \texttt{ggsave()}
\end{itemize}

\subsection{Key takeaways}\label{key-takeaways}

\begin{enumerate}
\def\labelenumi{\arabic{enumi}.}
\tightlist
\item
  \textbf{Always visualize your data} --- summary stats can hide
  patterns
\item
  \textbf{ggplot2 uses layers} --- data + aesthetics + geoms
\item
  \textbf{Aesthetics can be mapped or set} --- \texttt{aes()} vs direct
  values
\item
  \textbf{Facets are powerful} --- split plots by categories
\item
  \textbf{Labels and themes matter} --- make your plots readable
\end{enumerate}

\subsection{Questions?}\label{questions}

Next time: \textbf{Data Transformation with dplyr}

We'll learn to filter, sort, and create new variables!




\end{document}
