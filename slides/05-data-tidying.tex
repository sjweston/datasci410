% Options for packages loaded elsewhere
% Options for packages loaded elsewhere
\PassOptionsToPackage{unicode}{hyperref}
\PassOptionsToPackage{hyphens}{url}
\PassOptionsToPackage{dvipsnames,svgnames,x11names}{xcolor}
%
\documentclass[
  letterpaper,
  DIV=11,
  numbers=noendperiod]{scrartcl}
\usepackage{xcolor}
\usepackage{amsmath,amssymb}
\setcounter{secnumdepth}{-\maxdimen} % remove section numbering
\usepackage{iftex}
\ifPDFTeX
  \usepackage[T1]{fontenc}
  \usepackage[utf8]{inputenc}
  \usepackage{textcomp} % provide euro and other symbols
\else % if luatex or xetex
  \usepackage{unicode-math} % this also loads fontspec
  \defaultfontfeatures{Scale=MatchLowercase}
  \defaultfontfeatures[\rmfamily]{Ligatures=TeX,Scale=1}
\fi
\usepackage{lmodern}
\ifPDFTeX\else
  % xetex/luatex font selection
\fi
% Use upquote if available, for straight quotes in verbatim environments
\IfFileExists{upquote.sty}{\usepackage{upquote}}{}
\IfFileExists{microtype.sty}{% use microtype if available
  \usepackage[]{microtype}
  \UseMicrotypeSet[protrusion]{basicmath} % disable protrusion for tt fonts
}{}
\makeatletter
\@ifundefined{KOMAClassName}{% if non-KOMA class
  \IfFileExists{parskip.sty}{%
    \usepackage{parskip}
  }{% else
    \setlength{\parindent}{0pt}
    \setlength{\parskip}{6pt plus 2pt minus 1pt}}
}{% if KOMA class
  \KOMAoptions{parskip=half}}
\makeatother
% Make \paragraph and \subparagraph free-standing
\makeatletter
\ifx\paragraph\undefined\else
  \let\oldparagraph\paragraph
  \renewcommand{\paragraph}{
    \@ifstar
      \xxxParagraphStar
      \xxxParagraphNoStar
  }
  \newcommand{\xxxParagraphStar}[1]{\oldparagraph*{#1}\mbox{}}
  \newcommand{\xxxParagraphNoStar}[1]{\oldparagraph{#1}\mbox{}}
\fi
\ifx\subparagraph\undefined\else
  \let\oldsubparagraph\subparagraph
  \renewcommand{\subparagraph}{
    \@ifstar
      \xxxSubParagraphStar
      \xxxSubParagraphNoStar
  }
  \newcommand{\xxxSubParagraphStar}[1]{\oldsubparagraph*{#1}\mbox{}}
  \newcommand{\xxxSubParagraphNoStar}[1]{\oldsubparagraph{#1}\mbox{}}
\fi
\makeatother

\usepackage{color}
\usepackage{fancyvrb}
\newcommand{\VerbBar}{|}
\newcommand{\VERB}{\Verb[commandchars=\\\{\}]}
\DefineVerbatimEnvironment{Highlighting}{Verbatim}{commandchars=\\\{\}}
% Add ',fontsize=\small' for more characters per line
\usepackage{framed}
\definecolor{shadecolor}{RGB}{241,243,245}
\newenvironment{Shaded}{\begin{snugshade}}{\end{snugshade}}
\newcommand{\AlertTok}[1]{\textcolor[rgb]{0.68,0.00,0.00}{#1}}
\newcommand{\AnnotationTok}[1]{\textcolor[rgb]{0.37,0.37,0.37}{#1}}
\newcommand{\AttributeTok}[1]{\textcolor[rgb]{0.40,0.45,0.13}{#1}}
\newcommand{\BaseNTok}[1]{\textcolor[rgb]{0.68,0.00,0.00}{#1}}
\newcommand{\BuiltInTok}[1]{\textcolor[rgb]{0.00,0.23,0.31}{#1}}
\newcommand{\CharTok}[1]{\textcolor[rgb]{0.13,0.47,0.30}{#1}}
\newcommand{\CommentTok}[1]{\textcolor[rgb]{0.37,0.37,0.37}{#1}}
\newcommand{\CommentVarTok}[1]{\textcolor[rgb]{0.37,0.37,0.37}{\textit{#1}}}
\newcommand{\ConstantTok}[1]{\textcolor[rgb]{0.56,0.35,0.01}{#1}}
\newcommand{\ControlFlowTok}[1]{\textcolor[rgb]{0.00,0.23,0.31}{\textbf{#1}}}
\newcommand{\DataTypeTok}[1]{\textcolor[rgb]{0.68,0.00,0.00}{#1}}
\newcommand{\DecValTok}[1]{\textcolor[rgb]{0.68,0.00,0.00}{#1}}
\newcommand{\DocumentationTok}[1]{\textcolor[rgb]{0.37,0.37,0.37}{\textit{#1}}}
\newcommand{\ErrorTok}[1]{\textcolor[rgb]{0.68,0.00,0.00}{#1}}
\newcommand{\ExtensionTok}[1]{\textcolor[rgb]{0.00,0.23,0.31}{#1}}
\newcommand{\FloatTok}[1]{\textcolor[rgb]{0.68,0.00,0.00}{#1}}
\newcommand{\FunctionTok}[1]{\textcolor[rgb]{0.28,0.35,0.67}{#1}}
\newcommand{\ImportTok}[1]{\textcolor[rgb]{0.00,0.46,0.62}{#1}}
\newcommand{\InformationTok}[1]{\textcolor[rgb]{0.37,0.37,0.37}{#1}}
\newcommand{\KeywordTok}[1]{\textcolor[rgb]{0.00,0.23,0.31}{\textbf{#1}}}
\newcommand{\NormalTok}[1]{\textcolor[rgb]{0.00,0.23,0.31}{#1}}
\newcommand{\OperatorTok}[1]{\textcolor[rgb]{0.37,0.37,0.37}{#1}}
\newcommand{\OtherTok}[1]{\textcolor[rgb]{0.00,0.23,0.31}{#1}}
\newcommand{\PreprocessorTok}[1]{\textcolor[rgb]{0.68,0.00,0.00}{#1}}
\newcommand{\RegionMarkerTok}[1]{\textcolor[rgb]{0.00,0.23,0.31}{#1}}
\newcommand{\SpecialCharTok}[1]{\textcolor[rgb]{0.37,0.37,0.37}{#1}}
\newcommand{\SpecialStringTok}[1]{\textcolor[rgb]{0.13,0.47,0.30}{#1}}
\newcommand{\StringTok}[1]{\textcolor[rgb]{0.13,0.47,0.30}{#1}}
\newcommand{\VariableTok}[1]{\textcolor[rgb]{0.07,0.07,0.07}{#1}}
\newcommand{\VerbatimStringTok}[1]{\textcolor[rgb]{0.13,0.47,0.30}{#1}}
\newcommand{\WarningTok}[1]{\textcolor[rgb]{0.37,0.37,0.37}{\textit{#1}}}

\usepackage{longtable,booktabs,array}
\usepackage{calc} % for calculating minipage widths
% Correct order of tables after \paragraph or \subparagraph
\usepackage{etoolbox}
\makeatletter
\patchcmd\longtable{\par}{\if@noskipsec\mbox{}\fi\par}{}{}
\makeatother
% Allow footnotes in longtable head/foot
\IfFileExists{footnotehyper.sty}{\usepackage{footnotehyper}}{\usepackage{footnote}}
\makesavenoteenv{longtable}
\usepackage{graphicx}
\makeatletter
\newsavebox\pandoc@box
\newcommand*\pandocbounded[1]{% scales image to fit in text height/width
  \sbox\pandoc@box{#1}%
  \Gscale@div\@tempa{\textheight}{\dimexpr\ht\pandoc@box+\dp\pandoc@box\relax}%
  \Gscale@div\@tempb{\linewidth}{\wd\pandoc@box}%
  \ifdim\@tempb\p@<\@tempa\p@\let\@tempa\@tempb\fi% select the smaller of both
  \ifdim\@tempa\p@<\p@\scalebox{\@tempa}{\usebox\pandoc@box}%
  \else\usebox{\pandoc@box}%
  \fi%
}
% Set default figure placement to htbp
\def\fps@figure{htbp}
\makeatother





\setlength{\emergencystretch}{3em} % prevent overfull lines

\providecommand{\tightlist}{%
  \setlength{\itemsep}{0pt}\setlength{\parskip}{0pt}}



 


\KOMAoption{captions}{tableheading}
\makeatletter
\@ifpackageloaded{caption}{}{\usepackage{caption}}
\AtBeginDocument{%
\ifdefined\contentsname
  \renewcommand*\contentsname{Table of contents}
\else
  \newcommand\contentsname{Table of contents}
\fi
\ifdefined\listfigurename
  \renewcommand*\listfigurename{List of Figures}
\else
  \newcommand\listfigurename{List of Figures}
\fi
\ifdefined\listtablename
  \renewcommand*\listtablename{List of Tables}
\else
  \newcommand\listtablename{List of Tables}
\fi
\ifdefined\figurename
  \renewcommand*\figurename{Figure}
\else
  \newcommand\figurename{Figure}
\fi
\ifdefined\tablename
  \renewcommand*\tablename{Table}
\else
  \newcommand\tablename{Table}
\fi
}
\@ifpackageloaded{float}{}{\usepackage{float}}
\floatstyle{ruled}
\@ifundefined{c@chapter}{\newfloat{codelisting}{h}{lop}}{\newfloat{codelisting}{h}{lop}[chapter]}
\floatname{codelisting}{Listing}
\newcommand*\listoflistings{\listof{codelisting}{List of Listings}}
\makeatother
\makeatletter
\makeatother
\makeatletter
\@ifpackageloaded{caption}{}{\usepackage{caption}}
\@ifpackageloaded{subcaption}{}{\usepackage{subcaption}}
\makeatother
\usepackage{bookmark}
\IfFileExists{xurl.sty}{\usepackage{xurl}}{} % add URL line breaks if available
\urlstyle{same}
\hypersetup{
  pdftitle={Data Tidying},
  pdfauthor={Dr.~Sara Weston},
  colorlinks=true,
  linkcolor={blue},
  filecolor={Maroon},
  citecolor={Blue},
  urlcolor={Blue},
  pdfcreator={LaTeX via pandoc}}


\title{Data Tidying}
\usepackage{etoolbox}
\makeatletter
\providecommand{\subtitle}[1]{% add subtitle to \maketitle
  \apptocmd{\@title}{\par {\large #1 \par}}{}{}
}
\makeatother
\subtitle{PSY 410: Data Science for Psychology}
\author{Dr.~Sara Weston}
\date{2026-04-13}
\begin{document}
\maketitle


\section{What is tidy data?}\label{what-is-tidy-data}

\subsection{The tidyverse philosophy}\label{the-tidyverse-philosophy}

\begin{quote}
``Tidy datasets are all alike, but every messy dataset is messy in its
own way.'' --- Hadley Wickham
\end{quote}

The tools we've learned (ggplot2, dplyr) expect data in a specific
format: \textbf{tidy data}.

\subsection{The three rules of tidy
data}\label{the-three-rules-of-tidy-data}

\begin{enumerate}
\def\labelenumi{\arabic{enumi}.}
\tightlist
\item
  Each \textbf{variable} is a column
\item
  Each \textbf{observation} is a row
\item
  Each \textbf{value} is a cell
\end{enumerate}

. . .

Simple in theory, surprisingly complex in practice.

\subsection{Tidy data visualized}\label{tidy-data-visualized}

\begin{longtable}[]{@{}lrr@{}}
\caption{Is this tidy?}\tabularnewline
\toprule\noalign{}
participant & pre\_test & post\_test \\
\midrule\noalign{}
\endfirsthead
\toprule\noalign{}
participant & pre\_test & post\_test \\
\midrule\noalign{}
\endhead
\bottomrule\noalign{}
\endlastfoot
P1 & 45 & 62 \\
P2 & 52 & 58 \\
P3 & 48 & 71 \\
\end{longtable}

. . .

\textbf{No!} Time (pre/post) is a variable, but it's spread across
columns.

\subsection{The tidy version}\label{the-tidy-version}

\begin{longtable}[]{@{}llr@{}}
\caption{Now it's tidy!}\tabularnewline
\toprule\noalign{}
participant & time & score \\
\midrule\noalign{}
\endfirsthead
\toprule\noalign{}
participant & time & score \\
\midrule\noalign{}
\endhead
\bottomrule\noalign{}
\endlastfoot
P1 & pre & 45 \\
P1 & post & 62 \\
P2 & pre & 52 \\
P2 & post & 58 \\
P3 & pre & 48 \\
P3 & post & 71 \\
\end{longtable}

Now: participant, time, and score are all columns.

\subsection{Why does it matter?}\label{why-does-it-matter}

\textbf{Tidy data works with tidyverse tools:}

\begin{Shaded}
\begin{Highlighting}[]
\CommentTok{\# Easy to visualize}
\FunctionTok{ggplot}\NormalTok{(data, }\FunctionTok{aes}\NormalTok{(}\AttributeTok{x =}\NormalTok{ time, }\AttributeTok{y =}\NormalTok{ score, }\AttributeTok{color =}\NormalTok{ participant)) }\SpecialCharTok{+}
  \FunctionTok{geom\_point}\NormalTok{()}

\CommentTok{\# Easy to analyze}
\NormalTok{data }\SpecialCharTok{|\textgreater{}}
  \FunctionTok{group\_by}\NormalTok{(time) }\SpecialCharTok{|\textgreater{}}
  \FunctionTok{summarize}\NormalTok{(}\AttributeTok{mean =} \FunctionTok{mean}\NormalTok{(score))}
\end{Highlighting}
\end{Shaded}

\subsection{Wide vs.~Long}\label{wide-vs.-long}

\textbf{Wide format}

\begin{itemize}
\tightlist
\item
  Variables spread across columns
\item
  One row per subject
\item
  Humans like to read this
\end{itemize}

\textbf{Long format}

\begin{itemize}
\tightlist
\item
  Variables in a single column
\item
  Multiple rows per subject
\item
  R likes to work with this
\end{itemize}

Most real data needs reshaping.

\section{Common untidy patterns}\label{common-untidy-patterns}

\subsection{Pattern 1: Column headers are
values}\label{pattern-1-column-headers-are-values}

\begin{Shaded}
\begin{Highlighting}[]
\CommentTok{\# Exam scores by semester}
\NormalTok{wide\_scores }\OtherTok{\textless{}{-}} \FunctionTok{tibble}\NormalTok{(}
  \AttributeTok{student =} \FunctionTok{c}\NormalTok{(}\StringTok{"Alice"}\NormalTok{, }\StringTok{"Bob"}\NormalTok{, }\StringTok{"Carol"}\NormalTok{),}
  \AttributeTok{fall\_2024 =} \FunctionTok{c}\NormalTok{(}\DecValTok{85}\NormalTok{, }\DecValTok{78}\NormalTok{, }\DecValTok{92}\NormalTok{),}
  \AttributeTok{spring\_2025 =} \FunctionTok{c}\NormalTok{(}\DecValTok{88}\NormalTok{, }\DecValTok{82}\NormalTok{, }\DecValTok{95}\NormalTok{),}
  \AttributeTok{fall\_2025 =} \FunctionTok{c}\NormalTok{(}\DecValTok{91}\NormalTok{, }\DecValTok{85}\NormalTok{, }\DecValTok{94}\NormalTok{)}
\NormalTok{)}
\NormalTok{wide\_scores}
\end{Highlighting}
\end{Shaded}

\begin{verbatim}
# A tibble: 3 x 4
  student fall_2024 spring_2025 fall_2025
  <chr>       <dbl>       <dbl>     <dbl>
1 Alice          85          88        91
2 Bob            78          82        85
3 Carol          92          95        94
\end{verbatim}

The semester names are \textbf{values}, not variable names.

\subsection{Pattern 2: Multiple variables in one
column}\label{pattern-2-multiple-variables-in-one-column}

\begin{Shaded}
\begin{Highlighting}[]
\CommentTok{\# Data export with combined columns}
\NormalTok{messy\_data }\OtherTok{\textless{}{-}} \FunctionTok{tibble}\NormalTok{(}
  \AttributeTok{id =} \DecValTok{1}\SpecialCharTok{:}\DecValTok{3}\NormalTok{,}
  \AttributeTok{age\_sex =} \FunctionTok{c}\NormalTok{(}\StringTok{"25\_M"}\NormalTok{, }\StringTok{"32\_F"}\NormalTok{, }\StringTok{"28\_F"}\NormalTok{)}
\NormalTok{)}
\NormalTok{messy\_data}
\end{Highlighting}
\end{Shaded}

\begin{verbatim}
# A tibble: 3 x 2
     id age_sex
  <int> <chr>  
1     1 25_M   
2     2 32_F   
3     3 28_F   
\end{verbatim}

Age and sex are crammed into one column.

\subsection{Pattern 3: Variables in rows and
columns}\label{pattern-3-variables-in-rows-and-columns}

\begin{Shaded}
\begin{Highlighting}[]
\CommentTok{\# Weather data}
\NormalTok{weather }\OtherTok{\textless{}{-}} \FunctionTok{tibble}\NormalTok{(}
  \AttributeTok{id =} \FunctionTok{c}\NormalTok{(}\StringTok{"MX001"}\NormalTok{, }\StringTok{"MX001"}\NormalTok{, }\StringTok{"MX002"}\NormalTok{, }\StringTok{"MX002"}\NormalTok{),}
  \AttributeTok{year =} \FunctionTok{c}\NormalTok{(}\DecValTok{2020}\NormalTok{, }\DecValTok{2020}\NormalTok{, }\DecValTok{2020}\NormalTok{, }\DecValTok{2020}\NormalTok{),}
  \AttributeTok{month =} \FunctionTok{c}\NormalTok{(}\DecValTok{1}\NormalTok{, }\DecValTok{2}\NormalTok{, }\DecValTok{1}\NormalTok{, }\DecValTok{2}\NormalTok{),}
  \AttributeTok{element =} \FunctionTok{c}\NormalTok{(}\StringTok{"tmax"}\NormalTok{, }\StringTok{"tmax"}\NormalTok{, }\StringTok{"tmin"}\NormalTok{, }\StringTok{"tmin"}\NormalTok{),}
  \AttributeTok{value =} \FunctionTok{c}\NormalTok{(}\DecValTok{85}\NormalTok{, }\DecValTok{87}\NormalTok{, }\DecValTok{32}\NormalTok{, }\DecValTok{35}\NormalTok{)}
\NormalTok{)}
\NormalTok{weather}
\end{Highlighting}
\end{Shaded}

\begin{verbatim}
# A tibble: 4 x 5
  id     year month element value
  <chr> <dbl> <dbl> <chr>   <dbl>
1 MX001  2020     1 tmax       85
2 MX001  2020     2 tmax       87
3 MX002  2020     1 tmin       32
4 MX002  2020     2 tmin       35
\end{verbatim}

\texttt{element} contains variable names (tmax, tmin).

\subsection{Psychology-specific
patterns}\label{psychology-specific-patterns}

Surveys often look like:

\begin{Shaded}
\begin{Highlighting}[]
\CommentTok{\# Wide survey format}
\NormalTok{survey\_wide }\OtherTok{\textless{}{-}} \FunctionTok{tibble}\NormalTok{(}
  \AttributeTok{participant =} \DecValTok{1}\SpecialCharTok{:}\DecValTok{3}\NormalTok{,}
  \AttributeTok{bdi\_1 =} \FunctionTok{c}\NormalTok{(}\DecValTok{2}\NormalTok{, }\DecValTok{1}\NormalTok{, }\DecValTok{3}\NormalTok{),}
  \AttributeTok{bdi\_2 =} \FunctionTok{c}\NormalTok{(}\DecValTok{1}\NormalTok{, }\DecValTok{0}\NormalTok{, }\DecValTok{2}\NormalTok{),}
  \AttributeTok{bdi\_3 =} \FunctionTok{c}\NormalTok{(}\DecValTok{3}\NormalTok{, }\DecValTok{2}\NormalTok{, }\DecValTok{2}\NormalTok{),}
  \AttributeTok{bdi\_4 =} \FunctionTok{c}\NormalTok{(}\DecValTok{2}\NormalTok{, }\DecValTok{1}\NormalTok{, }\DecValTok{1}\NormalTok{)}
\NormalTok{)}
\NormalTok{survey\_wide}
\end{Highlighting}
\end{Shaded}

\begin{verbatim}
# A tibble: 3 x 5
  participant bdi_1 bdi_2 bdi_3 bdi_4
        <int> <dbl> <dbl> <dbl> <dbl>
1           1     2     1     3     2
2           2     1     0     2     1
3           3     3     2     2     1
\end{verbatim}

Each item is a column --- wide format.

\section{pivot\_longer()}\label{pivot_longer}

\subsection{The most common tidying
operation}\label{the-most-common-tidying-operation}

\texttt{pivot\_longer()} takes wide data and makes it long:

\begin{Shaded}
\begin{Highlighting}[]
\NormalTok{wide\_scores }\SpecialCharTok{|\textgreater{}}
  \FunctionTok{pivot\_longer}\NormalTok{(}
    \AttributeTok{cols =}\NormalTok{ fall\_2024}\SpecialCharTok{:}\NormalTok{fall\_2025,  }\CommentTok{\# Which columns to pivot}
    \AttributeTok{names\_to =} \StringTok{"semester"}\NormalTok{,        }\CommentTok{\# New column for old column names}
    \AttributeTok{values\_to =} \StringTok{"score"}           \CommentTok{\# New column for values}
\NormalTok{  )}
\end{Highlighting}
\end{Shaded}

\begin{verbatim}
# A tibble: 9 x 3
  student semester    score
  <chr>   <chr>       <dbl>
1 Alice   fall_2024      85
2 Alice   spring_2025    88
3 Alice   fall_2025      91
4 Bob     fall_2024      78
5 Bob     spring_2025    82
6 Bob     fall_2025      85
7 Carol   fall_2024      92
8 Carol   spring_2025    95
9 Carol   fall_2025      94
\end{verbatim}

\subsection{Breaking it down}\label{breaking-it-down}

\begin{Shaded}
\begin{Highlighting}[]
\FunctionTok{pivot\_longer}\NormalTok{(}
  \AttributeTok{cols =}\NormalTok{ ...,        }\CommentTok{\# Columns to reshape (use select helpers!)}
  \AttributeTok{names\_to =} \StringTok{"..."}\NormalTok{,  }\CommentTok{\# Name for the new "names" column}
  \AttributeTok{values\_to =} \StringTok{"..."}  \CommentTok{\# Name for the new "values" column}
\NormalTok{)}
\end{Highlighting}
\end{Shaded}

\subsection{Selecting columns to
pivot}\label{selecting-columns-to-pivot}

Use any of the \texttt{select()} helpers:

\begin{Shaded}
\begin{Highlighting}[]
\CommentTok{\# By name}
\FunctionTok{pivot\_longer}\NormalTok{(}\AttributeTok{cols =} \FunctionTok{c}\NormalTok{(fall\_2024, spring\_2025, fall\_2025))}

\CommentTok{\# By range}
\FunctionTok{pivot\_longer}\NormalTok{(}\AttributeTok{cols =}\NormalTok{ fall\_2024}\SpecialCharTok{:}\NormalTok{fall\_2025)}

\CommentTok{\# By pattern}
\FunctionTok{pivot\_longer}\NormalTok{(}\AttributeTok{cols =} \FunctionTok{starts\_with}\NormalTok{(}\StringTok{"fall"}\NormalTok{))}
\FunctionTok{pivot\_longer}\NormalTok{(}\AttributeTok{cols =} \FunctionTok{contains}\NormalTok{(}\StringTok{"202"}\NormalTok{))}

\CommentTok{\# Everything except}
\FunctionTok{pivot\_longer}\NormalTok{(}\AttributeTok{cols =} \SpecialCharTok{{-}}\NormalTok{student)}
\end{Highlighting}
\end{Shaded}

\subsection{Psychology example: Survey
items}\label{psychology-example-survey-items}

\begin{Shaded}
\begin{Highlighting}[]
\NormalTok{survey\_wide }\SpecialCharTok{|\textgreater{}}
  \FunctionTok{pivot\_longer}\NormalTok{(}
    \AttributeTok{cols =} \FunctionTok{starts\_with}\NormalTok{(}\StringTok{"bdi"}\NormalTok{),}
    \AttributeTok{names\_to =} \StringTok{"item"}\NormalTok{,}
    \AttributeTok{values\_to =} \StringTok{"response"}
\NormalTok{  )}
\end{Highlighting}
\end{Shaded}

\begin{verbatim}
# A tibble: 12 x 3
   participant item  response
         <int> <chr>    <dbl>
 1           1 bdi_1        2
 2           1 bdi_2        1
 3           1 bdi_3        3
 4           1 bdi_4        2
 5           2 bdi_1        1
 6           2 bdi_2        0
 7           2 bdi_3        2
 8           2 bdi_4        1
 9           3 bdi_1        3
10           3 bdi_2        2
11           3 bdi_3        2
12           3 bdi_4        1
\end{verbatim}

Now each response is its own row!

\subsection{Extracting information from
names}\label{extracting-information-from-names}

What if column names contain useful info?

\begin{Shaded}
\begin{Highlighting}[]
\CommentTok{\# Scores at different time points}
\NormalTok{experiment\_wide }\OtherTok{\textless{}{-}} \FunctionTok{tibble}\NormalTok{(}
  \AttributeTok{id =} \DecValTok{1}\SpecialCharTok{:}\DecValTok{3}\NormalTok{,}
  \AttributeTok{score\_t1 =} \FunctionTok{c}\NormalTok{(}\DecValTok{100}\NormalTok{, }\DecValTok{95}\NormalTok{, }\DecValTok{110}\NormalTok{),}
  \AttributeTok{score\_t2 =} \FunctionTok{c}\NormalTok{(}\DecValTok{105}\NormalTok{, }\DecValTok{100}\NormalTok{, }\DecValTok{115}\NormalTok{),}
  \AttributeTok{score\_t3 =} \FunctionTok{c}\NormalTok{(}\DecValTok{108}\NormalTok{, }\DecValTok{102}\NormalTok{, }\DecValTok{120}\NormalTok{)}
\NormalTok{)}
\NormalTok{experiment\_wide}
\end{Highlighting}
\end{Shaded}

\begin{verbatim}
# A tibble: 3 x 4
     id score_t1 score_t2 score_t3
  <int>    <dbl>    <dbl>    <dbl>
1     1      100      105      108
2     2       95      100      102
3     3      110      115      120
\end{verbatim}

We want to extract the time point (t1, t2, t3).

\subsection{names\_prefix argument}\label{names_prefix-argument}

\begin{Shaded}
\begin{Highlighting}[]
\NormalTok{experiment\_wide }\SpecialCharTok{|\textgreater{}}
  \FunctionTok{pivot\_longer}\NormalTok{(}
    \AttributeTok{cols =} \FunctionTok{starts\_with}\NormalTok{(}\StringTok{"score"}\NormalTok{),}
    \AttributeTok{names\_to =} \StringTok{"time"}\NormalTok{,}
    \AttributeTok{names\_prefix =} \StringTok{"score\_"}\NormalTok{,  }\CommentTok{\# Remove this prefix from names}
    \AttributeTok{values\_to =} \StringTok{"score"}
\NormalTok{  )}
\end{Highlighting}
\end{Shaded}

\begin{verbatim}
# A tibble: 9 x 3
     id time  score
  <int> <chr> <dbl>
1     1 t1      100
2     1 t2      105
3     1 t3      108
4     2 t1       95
5     2 t2      100
6     2 t3      102
7     3 t1      110
8     3 t2      115
9     3 t3      120
\end{verbatim}

\subsection{names\_pattern argument}\label{names_pattern-argument}

For more complex parsing:

\begin{Shaded}
\begin{Highlighting}[]
\CommentTok{\# Column names like "bdi\_1", "anxiety\_1", etc.}
\NormalTok{multi\_scale }\OtherTok{\textless{}{-}} \FunctionTok{tibble}\NormalTok{(}
  \AttributeTok{id =} \DecValTok{1}\SpecialCharTok{:}\DecValTok{2}\NormalTok{,}
  \AttributeTok{bdi\_1 =} \FunctionTok{c}\NormalTok{(}\DecValTok{2}\NormalTok{, }\DecValTok{1}\NormalTok{), }\AttributeTok{bdi\_2 =} \FunctionTok{c}\NormalTok{(}\DecValTok{1}\NormalTok{, }\DecValTok{2}\NormalTok{),}
  \AttributeTok{anxiety\_1 =} \FunctionTok{c}\NormalTok{(}\DecValTok{3}\NormalTok{, }\DecValTok{2}\NormalTok{), }\AttributeTok{anxiety\_2 =} \FunctionTok{c}\NormalTok{(}\DecValTok{2}\NormalTok{, }\DecValTok{3}\NormalTok{)}
\NormalTok{)}

\NormalTok{multi\_scale }\SpecialCharTok{|\textgreater{}}
  \FunctionTok{pivot\_longer}\NormalTok{(}
    \AttributeTok{cols =} \SpecialCharTok{{-}}\NormalTok{id,}
    \AttributeTok{names\_to =} \FunctionTok{c}\NormalTok{(}\StringTok{"scale"}\NormalTok{, }\StringTok{"item"}\NormalTok{),}
    \AttributeTok{names\_pattern =} \StringTok{"(.+)\_(.+)"}\NormalTok{,  }\CommentTok{\# Regex: anything\_anything}
    \AttributeTok{values\_to =} \StringTok{"response"}
\NormalTok{  )}
\end{Highlighting}
\end{Shaded}

\begin{verbatim}
# A tibble: 8 x 4
     id scale   item  response
  <int> <chr>   <chr>    <dbl>
1     1 bdi     1            2
2     1 bdi     2            1
3     1 anxiety 1            3
4     1 anxiety 2            2
5     2 bdi     1            1
6     2 bdi     2            2
7     2 anxiety 1            2
8     2 anxiety 2            3
\end{verbatim}

\subsection{Pivoting to calculate scale
scores}\label{pivoting-to-calculate-scale-scores}

\begin{Shaded}
\begin{Highlighting}[]
\CommentTok{\# Calculate BDI total from long format}
\NormalTok{survey\_wide }\SpecialCharTok{|\textgreater{}}
  \FunctionTok{pivot\_longer}\NormalTok{(}
    \AttributeTok{cols =} \FunctionTok{starts\_with}\NormalTok{(}\StringTok{"bdi"}\NormalTok{),}
    \AttributeTok{names\_to =} \StringTok{"item"}\NormalTok{,}
    \AttributeTok{values\_to =} \StringTok{"response"}
\NormalTok{  ) }\SpecialCharTok{|\textgreater{}}
  \FunctionTok{group\_by}\NormalTok{(participant) }\SpecialCharTok{|\textgreater{}}
  \FunctionTok{summarize}\NormalTok{(}\AttributeTok{bdi\_total =} \FunctionTok{sum}\NormalTok{(response))}
\end{Highlighting}
\end{Shaded}

\begin{verbatim}
# A tibble: 3 x 2
  participant bdi_total
        <int>     <dbl>
1           1         8
2           2         4
3           3         8
\end{verbatim}

\section{pivot\_wider()}\label{pivot_wider}

\subsection{The opposite operation}\label{the-opposite-operation}

Sometimes you need to go from long to wide:

\begin{Shaded}
\begin{Highlighting}[]
\NormalTok{long\_data }\OtherTok{\textless{}{-}} \FunctionTok{tibble}\NormalTok{(}
  \AttributeTok{participant =} \FunctionTok{rep}\NormalTok{(}\DecValTok{1}\SpecialCharTok{:}\DecValTok{3}\NormalTok{, }\AttributeTok{each =} \DecValTok{2}\NormalTok{),}
  \AttributeTok{time =} \FunctionTok{rep}\NormalTok{(}\FunctionTok{c}\NormalTok{(}\StringTok{"pre"}\NormalTok{, }\StringTok{"post"}\NormalTok{), }\DecValTok{3}\NormalTok{),}
  \AttributeTok{score =} \FunctionTok{c}\NormalTok{(}\DecValTok{45}\NormalTok{, }\DecValTok{62}\NormalTok{, }\DecValTok{52}\NormalTok{, }\DecValTok{58}\NormalTok{, }\DecValTok{48}\NormalTok{, }\DecValTok{71}\NormalTok{)}
\NormalTok{)}
\NormalTok{long\_data}
\end{Highlighting}
\end{Shaded}

\begin{verbatim}
# A tibble: 6 x 3
  participant time  score
        <int> <chr> <dbl>
1           1 pre      45
2           1 post     62
3           2 pre      52
4           2 post     58
5           3 pre      48
6           3 post     71
\end{verbatim}

\subsection{pivot\_wider() syntax}\label{pivot_wider-syntax}

\begin{Shaded}
\begin{Highlighting}[]
\NormalTok{long\_data }\SpecialCharTok{|\textgreater{}}
  \FunctionTok{pivot\_wider}\NormalTok{(}
    \AttributeTok{names\_from =}\NormalTok{ time,   }\CommentTok{\# Column to get new column names from}
    \AttributeTok{values\_from =}\NormalTok{ score  }\CommentTok{\# Column to get values from}
\NormalTok{  )}
\end{Highlighting}
\end{Shaded}

\begin{verbatim}
# A tibble: 3 x 3
  participant   pre  post
        <int> <dbl> <dbl>
1           1    45    62
2           2    52    58
3           3    48    71
\end{verbatim}

\subsection{When to use pivot\_wider()}\label{when-to-use-pivot_wider}

\begin{itemize}
\tightlist
\item
  Creating summary tables for reports
\item
  Some analyses need wide format
\item
  Merging data that was collected differently
\item
  Human-readable output
\end{itemize}

\subsection{Multiple value columns}\label{multiple-value-columns}

\begin{Shaded}
\begin{Highlighting}[]
\CommentTok{\# Long data with multiple measures}
\NormalTok{long\_multi }\OtherTok{\textless{}{-}} \FunctionTok{tibble}\NormalTok{(}
  \AttributeTok{id =} \FunctionTok{rep}\NormalTok{(}\DecValTok{1}\SpecialCharTok{:}\DecValTok{2}\NormalTok{, }\AttributeTok{each =} \DecValTok{2}\NormalTok{),}
  \AttributeTok{time =} \FunctionTok{rep}\NormalTok{(}\FunctionTok{c}\NormalTok{(}\StringTok{"pre"}\NormalTok{, }\StringTok{"post"}\NormalTok{), }\DecValTok{2}\NormalTok{),}
  \AttributeTok{score =} \FunctionTok{c}\NormalTok{(}\DecValTok{45}\NormalTok{, }\DecValTok{62}\NormalTok{, }\DecValTok{52}\NormalTok{, }\DecValTok{58}\NormalTok{),}
  \AttributeTok{rt =} \FunctionTok{c}\NormalTok{(}\DecValTok{500}\NormalTok{, }\DecValTok{480}\NormalTok{, }\DecValTok{520}\NormalTok{, }\DecValTok{490}\NormalTok{)}
\NormalTok{)}

\NormalTok{long\_multi }\SpecialCharTok{|\textgreater{}}
  \FunctionTok{pivot\_wider}\NormalTok{(}
    \AttributeTok{names\_from =}\NormalTok{ time,}
    \AttributeTok{values\_from =} \FunctionTok{c}\NormalTok{(score, rt)  }\CommentTok{\# Multiple columns!}
\NormalTok{  )}
\end{Highlighting}
\end{Shaded}

\begin{verbatim}
# A tibble: 2 x 5
     id score_pre score_post rt_pre rt_post
  <int>     <dbl>      <dbl>  <dbl>   <dbl>
1     1        45         62    500     480
2     2        52         58    520     490
\end{verbatim}

\subsection{Creating an APA-style summary
table}\label{creating-an-apa-style-summary-table}

\begin{Shaded}
\begin{Highlighting}[]
\CommentTok{\# Start with your data}
\NormalTok{experiment }\OtherTok{\textless{}{-}} \FunctionTok{tibble}\NormalTok{(}
  \AttributeTok{id =} \DecValTok{1}\SpecialCharTok{:}\DecValTok{60}\NormalTok{,}
  \AttributeTok{condition =} \FunctionTok{rep}\NormalTok{(}\FunctionTok{c}\NormalTok{(}\StringTok{"control"}\NormalTok{, }\StringTok{"treatment"}\NormalTok{), }\DecValTok{30}\NormalTok{),}
  \AttributeTok{score =} \FunctionTok{c}\NormalTok{(}\FunctionTok{rnorm}\NormalTok{(}\DecValTok{30}\NormalTok{, }\DecValTok{50}\NormalTok{, }\DecValTok{10}\NormalTok{), }\FunctionTok{rnorm}\NormalTok{(}\DecValTok{30}\NormalTok{, }\DecValTok{55}\NormalTok{, }\DecValTok{10}\NormalTok{))}
\NormalTok{)}

\CommentTok{\# Create summary}
\NormalTok{experiment }\SpecialCharTok{|\textgreater{}}
  \FunctionTok{group\_by}\NormalTok{(condition) }\SpecialCharTok{|\textgreater{}}
  \FunctionTok{summarize}\NormalTok{(}
    \AttributeTok{M =} \FunctionTok{round}\NormalTok{(}\FunctionTok{mean}\NormalTok{(score), }\DecValTok{2}\NormalTok{),}
    \AttributeTok{SD =} \FunctionTok{round}\NormalTok{(}\FunctionTok{sd}\NormalTok{(score), }\DecValTok{2}\NormalTok{)}
\NormalTok{  ) }\SpecialCharTok{|\textgreater{}}
  \FunctionTok{pivot\_wider}\NormalTok{(}
    \AttributeTok{names\_from =}\NormalTok{ condition,}
    \AttributeTok{values\_from =} \FunctionTok{c}\NormalTok{(M, SD)}
\NormalTok{  )}
\end{Highlighting}
\end{Shaded}

\begin{verbatim}
# A tibble: 1 x 4
  M_control M_treatment SD_control SD_treatment
      <dbl>       <dbl>      <dbl>        <dbl>
1      53.5        51.0       9.22         11.2
\end{verbatim}

\section{Separating and uniting}\label{separating-and-uniting}

\subsection{separate\_wider\_delim()}\label{separate_wider_delim}

Split one column into multiple:

\begin{Shaded}
\begin{Highlighting}[]
\NormalTok{messy\_data }\OtherTok{\textless{}{-}} \FunctionTok{tibble}\NormalTok{(}
  \AttributeTok{id =} \DecValTok{1}\SpecialCharTok{:}\DecValTok{3}\NormalTok{,}
  \AttributeTok{age\_sex =} \FunctionTok{c}\NormalTok{(}\StringTok{"25\_M"}\NormalTok{, }\StringTok{"32\_F"}\NormalTok{, }\StringTok{"28\_F"}\NormalTok{)}
\NormalTok{)}

\NormalTok{messy\_data }\SpecialCharTok{|\textgreater{}}
  \FunctionTok{separate\_wider\_delim}\NormalTok{(}
    \AttributeTok{cols =}\NormalTok{ age\_sex,}
    \AttributeTok{delim =} \StringTok{"\_"}\NormalTok{,}
    \AttributeTok{names =} \FunctionTok{c}\NormalTok{(}\StringTok{"age"}\NormalTok{, }\StringTok{"sex"}\NormalTok{)}
\NormalTok{  )}
\end{Highlighting}
\end{Shaded}

\begin{verbatim}
# A tibble: 3 x 3
     id age   sex  
  <int> <chr> <chr>
1     1 25    M    
2     2 32    F    
3     3 28    F    
\end{verbatim}

\subsection{separate\_wider\_regex()}\label{separate_wider_regex}

For complex patterns:

\begin{Shaded}
\begin{Highlighting}[]
\FunctionTok{tibble}\NormalTok{(}
  \AttributeTok{code =} \FunctionTok{c}\NormalTok{(}\StringTok{"A123"}\NormalTok{, }\StringTok{"B456"}\NormalTok{, }\StringTok{"C789"}\NormalTok{)}
\NormalTok{) }\SpecialCharTok{|\textgreater{}}
  \FunctionTok{separate\_wider\_regex}\NormalTok{(}
    \AttributeTok{cols =}\NormalTok{ code,}
    \AttributeTok{patterns =} \FunctionTok{c}\NormalTok{(}\AttributeTok{letter =} \StringTok{"[A{-}Z]"}\NormalTok{, }\AttributeTok{number =} \StringTok{"[0{-}9]+"}\NormalTok{)}
\NormalTok{  )}
\end{Highlighting}
\end{Shaded}

\begin{verbatim}
# A tibble: 3 x 2
  letter number
  <chr>  <chr> 
1 A      123   
2 B      456   
3 C      789   
\end{verbatim}

\subsection{unite()}\label{unite}

The opposite --- combine columns:

\begin{Shaded}
\begin{Highlighting}[]
\FunctionTok{tibble}\NormalTok{(}
  \AttributeTok{year =} \FunctionTok{c}\NormalTok{(}\DecValTok{2024}\NormalTok{, }\DecValTok{2024}\NormalTok{, }\DecValTok{2025}\NormalTok{),}
  \AttributeTok{month =} \FunctionTok{c}\NormalTok{(}\DecValTok{1}\NormalTok{, }\DecValTok{6}\NormalTok{, }\DecValTok{1}\NormalTok{),}
  \AttributeTok{day =} \FunctionTok{c}\NormalTok{(}\DecValTok{15}\NormalTok{, }\DecValTok{20}\NormalTok{, }\DecValTok{10}\NormalTok{)}
\NormalTok{) }\SpecialCharTok{|\textgreater{}}
  \FunctionTok{unite}\NormalTok{(}
    \AttributeTok{col =} \StringTok{"date"}\NormalTok{,       }\CommentTok{\# New column name}
\NormalTok{    year, month, day,   }\CommentTok{\# Columns to combine}
    \AttributeTok{sep =} \StringTok{"{-}"}           \CommentTok{\# Separator}
\NormalTok{  )}
\end{Highlighting}
\end{Shaded}

\begin{verbatim}
# A tibble: 3 x 1
  date     
  <chr>    
1 2024-1-15
2 2024-6-20
3 2025-1-10
\end{verbatim}

\section{Real-world examples}\label{real-world-examples}

\subsection{Example 1: Repeated measures
experiment}\label{example-1-repeated-measures-experiment}

\begin{Shaded}
\begin{Highlighting}[]
\CommentTok{\# Data as you might receive it from SPSS}
\NormalTok{wide\_rm }\OtherTok{\textless{}{-}} \FunctionTok{tibble}\NormalTok{(}
  \AttributeTok{subject =} \DecValTok{1}\SpecialCharTok{:}\DecValTok{4}\NormalTok{,}
  \AttributeTok{cond\_a\_time1 =} \FunctionTok{c}\NormalTok{(}\DecValTok{450}\NormalTok{, }\DecValTok{520}\NormalTok{, }\DecValTok{480}\NormalTok{, }\DecValTok{510}\NormalTok{),}
  \AttributeTok{cond\_a\_time2 =} \FunctionTok{c}\NormalTok{(}\DecValTok{420}\NormalTok{, }\DecValTok{490}\NormalTok{, }\DecValTok{460}\NormalTok{, }\DecValTok{480}\NormalTok{),}
  \AttributeTok{cond\_b\_time1 =} \FunctionTok{c}\NormalTok{(}\DecValTok{480}\NormalTok{, }\DecValTok{540}\NormalTok{, }\DecValTok{500}\NormalTok{, }\DecValTok{530}\NormalTok{),}
  \AttributeTok{cond\_b\_time2 =} \FunctionTok{c}\NormalTok{(}\DecValTok{440}\NormalTok{, }\DecValTok{510}\NormalTok{, }\DecValTok{470}\NormalTok{, }\DecValTok{500}\NormalTok{)}
\NormalTok{)}
\NormalTok{wide\_rm}
\end{Highlighting}
\end{Shaded}

\begin{verbatim}
# A tibble: 4 x 5
  subject cond_a_time1 cond_a_time2 cond_b_time1 cond_b_time2
    <int>        <dbl>        <dbl>        <dbl>        <dbl>
1       1          450          420          480          440
2       2          520          490          540          510
3       3          480          460          500          470
4       4          510          480          530          500
\end{verbatim}

\subsection{Tidying repeated measures}\label{tidying-repeated-measures}

\begin{Shaded}
\begin{Highlighting}[]
\NormalTok{tidy\_rm }\OtherTok{\textless{}{-}}\NormalTok{ wide\_rm }\SpecialCharTok{|\textgreater{}}
  \FunctionTok{pivot\_longer}\NormalTok{(}
    \AttributeTok{cols =} \SpecialCharTok{{-}}\NormalTok{subject,}
    \AttributeTok{names\_to =} \FunctionTok{c}\NormalTok{(}\StringTok{"condition"}\NormalTok{, }\StringTok{"time"}\NormalTok{),}
    \AttributeTok{names\_pattern =} \StringTok{"cond\_(.+)\_time(.+)"}\NormalTok{,}
    \AttributeTok{values\_to =} \StringTok{"rt"}
\NormalTok{  )}
\NormalTok{tidy\_rm}
\end{Highlighting}
\end{Shaded}

\begin{verbatim}
# A tibble: 16 x 4
   subject condition time     rt
     <int> <chr>     <chr> <dbl>
 1       1 a         1       450
 2       1 a         2       420
 3       1 b         1       480
 4       1 b         2       440
 5       2 a         1       520
 6       2 a         2       490
 7       2 b         1       540
 8       2 b         2       510
 9       3 a         1       480
10       3 a         2       460
11       3 b         1       500
12       3 b         2       470
13       4 a         1       510
14       4 a         2       480
15       4 b         1       530
16       4 b         2       500
\end{verbatim}

\subsection{Now we can analyze it!}\label{now-we-can-analyze-it}

\begin{Shaded}
\begin{Highlighting}[]
\NormalTok{tidy\_rm }\SpecialCharTok{|\textgreater{}}
  \FunctionTok{ggplot}\NormalTok{(}\FunctionTok{aes}\NormalTok{(}\AttributeTok{x =}\NormalTok{ time, }\AttributeTok{y =}\NormalTok{ rt, }\AttributeTok{color =}\NormalTok{ condition, }\AttributeTok{group =}\NormalTok{ condition)) }\SpecialCharTok{+}
  \FunctionTok{stat\_summary}\NormalTok{(}\AttributeTok{fun =}\NormalTok{ mean, }\AttributeTok{geom =} \StringTok{"point"}\NormalTok{, }\AttributeTok{size =} \DecValTok{3}\NormalTok{) }\SpecialCharTok{+}
  \FunctionTok{stat\_summary}\NormalTok{(}\AttributeTok{fun =}\NormalTok{ mean, }\AttributeTok{geom =} \StringTok{"line"}\NormalTok{) }\SpecialCharTok{+}
  \FunctionTok{stat\_summary}\NormalTok{(}\AttributeTok{fun.data =}\NormalTok{ mean\_se, }\AttributeTok{geom =} \StringTok{"errorbar"}\NormalTok{, }\AttributeTok{width =} \FloatTok{0.1}\NormalTok{) }\SpecialCharTok{+}
  \FunctionTok{labs}\NormalTok{(}
    \AttributeTok{title =} \StringTok{"Reaction Time by Condition and Time"}\NormalTok{,}
    \AttributeTok{x =} \StringTok{"Time Point"}\NormalTok{,}
    \AttributeTok{y =} \StringTok{"Reaction Time (ms)"}
\NormalTok{  ) }\SpecialCharTok{+}
  \FunctionTok{theme\_minimal}\NormalTok{(}\AttributeTok{base\_size =} \DecValTok{14}\NormalTok{)}
\end{Highlighting}
\end{Shaded}

\pandocbounded{\includegraphics[keepaspectratio]{05-data-tidying_files/figure-pdf/unnamed-chunk-26-1.pdf}}

\subsection{Example 2: Questionnaire with
subscales}\label{example-2-questionnaire-with-subscales}

\begin{Shaded}
\begin{Highlighting}[]
\CommentTok{\# Raw questionnaire data}
\NormalTok{quest }\OtherTok{\textless{}{-}} \FunctionTok{tibble}\NormalTok{(}
  \AttributeTok{pid =} \DecValTok{1}\SpecialCharTok{:}\DecValTok{3}\NormalTok{,}
  \AttributeTok{anx\_1 =} \FunctionTok{c}\NormalTok{(}\DecValTok{3}\NormalTok{, }\DecValTok{2}\NormalTok{, }\DecValTok{4}\NormalTok{), }\AttributeTok{anx\_2 =} \FunctionTok{c}\NormalTok{(}\DecValTok{2}\NormalTok{, }\DecValTok{3}\NormalTok{, }\DecValTok{3}\NormalTok{), }\AttributeTok{anx\_3 =} \FunctionTok{c}\NormalTok{(}\DecValTok{4}\NormalTok{, }\DecValTok{2}\NormalTok{, }\DecValTok{5}\NormalTok{),}
  \AttributeTok{dep\_1 =} \FunctionTok{c}\NormalTok{(}\DecValTok{2}\NormalTok{, }\DecValTok{1}\NormalTok{, }\DecValTok{3}\NormalTok{), }\AttributeTok{dep\_2 =} \FunctionTok{c}\NormalTok{(}\DecValTok{3}\NormalTok{, }\DecValTok{2}\NormalTok{, }\DecValTok{4}\NormalTok{), }\AttributeTok{dep\_3 =} \FunctionTok{c}\NormalTok{(}\DecValTok{2}\NormalTok{, }\DecValTok{1}\NormalTok{, }\DecValTok{3}\NormalTok{)}
\NormalTok{)}

\CommentTok{\# Tidy and calculate subscales}
\NormalTok{quest }\SpecialCharTok{|\textgreater{}}
  \FunctionTok{pivot\_longer}\NormalTok{(}
    \AttributeTok{cols =} \SpecialCharTok{{-}}\NormalTok{pid,}
    \AttributeTok{names\_to =} \FunctionTok{c}\NormalTok{(}\StringTok{"scale"}\NormalTok{, }\StringTok{"item"}\NormalTok{),}
    \AttributeTok{names\_pattern =} \StringTok{"(.+)\_(.+)"}\NormalTok{,}
    \AttributeTok{values\_to =} \StringTok{"response"}
\NormalTok{  ) }\SpecialCharTok{|\textgreater{}}
  \FunctionTok{group\_by}\NormalTok{(pid, scale) }\SpecialCharTok{|\textgreater{}}
  \FunctionTok{summarize}\NormalTok{(}\AttributeTok{subscale\_mean =} \FunctionTok{mean}\NormalTok{(response), }\AttributeTok{.groups =} \StringTok{"drop"}\NormalTok{) }\SpecialCharTok{|\textgreater{}}
  \FunctionTok{pivot\_wider}\NormalTok{(}\AttributeTok{names\_from =}\NormalTok{ scale, }\AttributeTok{values\_from =}\NormalTok{ subscale\_mean)}
\end{Highlighting}
\end{Shaded}

\begin{verbatim}
# A tibble: 3 x 3
    pid   anx   dep
  <int> <dbl> <dbl>
1     1  3     2.33
2     2  2.33  1.33
3     3  4     3.33
\end{verbatim}

\subsection{Example 3: Multilevel/nested
data}\label{example-3-multilevelnested-data}

\begin{Shaded}
\begin{Highlighting}[]
\CommentTok{\# Students nested in classrooms}
\NormalTok{students }\OtherTok{\textless{}{-}} \FunctionTok{tibble}\NormalTok{(}
  \AttributeTok{classroom =} \FunctionTok{rep}\NormalTok{(}\FunctionTok{c}\NormalTok{(}\StringTok{"A"}\NormalTok{, }\StringTok{"B"}\NormalTok{), }\AttributeTok{each =} \DecValTok{3}\NormalTok{),}
  \AttributeTok{student =} \DecValTok{1}\SpecialCharTok{:}\DecValTok{6}\NormalTok{,}
  \AttributeTok{pretest =} \FunctionTok{c}\NormalTok{(}\DecValTok{70}\NormalTok{, }\DecValTok{75}\NormalTok{, }\DecValTok{72}\NormalTok{, }\DecValTok{68}\NormalTok{, }\DecValTok{71}\NormalTok{, }\DecValTok{69}\NormalTok{),}
  \AttributeTok{posttest =} \FunctionTok{c}\NormalTok{(}\DecValTok{80}\NormalTok{, }\DecValTok{82}\NormalTok{, }\DecValTok{78}\NormalTok{, }\DecValTok{75}\NormalTok{, }\DecValTok{79}\NormalTok{, }\DecValTok{77}\NormalTok{)}
\NormalTok{)}

\CommentTok{\# Tidy for analysis}
\NormalTok{students }\SpecialCharTok{|\textgreater{}}
  \FunctionTok{pivot\_longer}\NormalTok{(}
    \AttributeTok{cols =} \FunctionTok{c}\NormalTok{(pretest, posttest),}
    \AttributeTok{names\_to =} \StringTok{"time"}\NormalTok{,}
    \AttributeTok{values\_to =} \StringTok{"score"}
\NormalTok{  )}
\end{Highlighting}
\end{Shaded}

\begin{verbatim}
# A tibble: 12 x 4
   classroom student time     score
   <chr>       <int> <chr>    <dbl>
 1 A               1 pretest     70
 2 A               1 posttest    80
 3 A               2 pretest     75
 4 A               2 posttest    82
 5 A               3 pretest     72
 6 A               3 posttest    78
 7 B               4 pretest     68
 8 B               4 posttest    75
 9 B               5 pretest     71
10 B               5 posttest    79
11 B               6 pretest     69
12 B               6 posttest    77
\end{verbatim}

\section{Common pitfalls}\label{common-pitfalls}

\subsection{Pitfall 1: Forgetting what's a
variable}\label{pitfall-1-forgetting-whats-a-variable}

Ask yourself: What are my \textbf{variables}?

\begin{itemize}
\tightlist
\item
  Participant ID? ✓ Variable
\item
  Time point? ✓ Variable (not separate columns!)
\item
  Score? ✓ Variable
\item
  Item number? Depends on your analysis
\end{itemize}

\subsection{Pitfall 2: Over-pivoting}\label{pitfall-2-over-pivoting}

Not everything needs to be long:

\begin{Shaded}
\begin{Highlighting}[]
\CommentTok{\# Maybe this is fine as{-}is?}
\FunctionTok{tibble}\NormalTok{(}
  \AttributeTok{id =} \DecValTok{1}\SpecialCharTok{:}\DecValTok{3}\NormalTok{,}
  \AttributeTok{age =} \FunctionTok{c}\NormalTok{(}\DecValTok{25}\NormalTok{, }\DecValTok{32}\NormalTok{, }\DecValTok{28}\NormalTok{),}
  \AttributeTok{gender =} \FunctionTok{c}\NormalTok{(}\StringTok{"M"}\NormalTok{, }\StringTok{"F"}\NormalTok{, }\StringTok{"F"}\NormalTok{),}
  \AttributeTok{score =} \FunctionTok{c}\NormalTok{(}\DecValTok{85}\NormalTok{, }\DecValTok{92}\NormalTok{, }\DecValTok{88}\NormalTok{)}
\NormalTok{)}
\end{Highlighting}
\end{Shaded}

Age, gender, and score are \textbf{different variables} --- keep them as
columns.

\subsection{Pitfall 3: Losing
information}\label{pitfall-3-losing-information}

Make sure your pivot preserves all data:

\begin{Shaded}
\begin{Highlighting}[]
\CommentTok{\# Before pivoting: 3 rows}
\NormalTok{wide\_scores}
\end{Highlighting}
\end{Shaded}

\begin{verbatim}
# A tibble: 3 x 4
  student fall_2024 spring_2025 fall_2025
  <chr>       <dbl>       <dbl>     <dbl>
1 Alice          85          88        91
2 Bob            78          82        85
3 Carol          92          95        94
\end{verbatim}

\begin{Shaded}
\begin{Highlighting}[]
\CommentTok{\# After pivoting: 9 rows (3 students × 3 semesters)}
\NormalTok{wide\_scores }\SpecialCharTok{|\textgreater{}}
  \FunctionTok{pivot\_longer}\NormalTok{(}\SpecialCharTok{{-}}\NormalTok{student, }\AttributeTok{names\_to =} \StringTok{"semester"}\NormalTok{, }\AttributeTok{values\_to =} \StringTok{"score"}\NormalTok{)}
\end{Highlighting}
\end{Shaded}

\begin{verbatim}
# A tibble: 9 x 3
  student semester    score
  <chr>   <chr>       <dbl>
1 Alice   fall_2024      85
2 Alice   spring_2025    88
3 Alice   fall_2025      91
4 Bob     fall_2024      78
5 Bob     spring_2025    82
6 Bob     fall_2025      85
7 Carol   fall_2024      92
8 Carol   spring_2025    95
9 Carol   fall_2025      94
\end{verbatim}

Check:
\texttt{nrow(original)\ ×\ ncol(pivoted\_columns)\ =\ nrow(result)}

\subsection{General tidying strategy}\label{general-tidying-strategy}

\begin{enumerate}
\def\labelenumi{\arabic{enumi}.}
\tightlist
\item
  \textbf{Identify} the variables (what are you measuring?)
\item
  \textbf{Look} at your current structure (what's a row? column?)
\item
  \textbf{Determine} what operations you need
\item
  \textbf{Test} with a small subset first
\item
  \textbf{Verify} you haven't lost data
\end{enumerate}

\section{Wrapping up}\label{wrapping-up}

\subsection{The tidyr toolkit}\label{the-tidyr-toolkit}

\begin{longtable}[]{@{}ll@{}}
\toprule\noalign{}
Function & What it does \\
\midrule\noalign{}
\endhead
\bottomrule\noalign{}
\endlastfoot
\texttt{pivot\_longer()} & Wide → Long \\
\texttt{pivot\_wider()} & Long → Wide \\
\texttt{separate\_*()} & Split columns \\
\texttt{unite()} & Combine columns \\
\end{longtable}

\subsection{The tidy data mantra}\label{the-tidy-data-mantra}

\begin{enumerate}
\def\labelenumi{\arabic{enumi}.}
\tightlist
\item
  Each \textbf{variable} is a column
\item
  Each \textbf{observation} is a row
\item
  Each \textbf{value} is a cell
\end{enumerate}

When in doubt, ask: ``What would make this easiest to plot/analyze?''

\subsection{Before next class}\label{before-next-class}

📖 \textbf{Read:}

\begin{itemize}
\tightlist
\item
  \href{https://r4ds.hadley.nz/data-import}{R4DS Ch 7: Data import}
\item
  \href{https://r4ds.hadley.nz/spreadsheets}{R4DS Ch 20: Spreadsheets}
\end{itemize}

✅ \textbf{Practice:}

\begin{itemize}
\tightlist
\item
  Reshape a dataset you've worked with
\item
  Try tidying some messy example data
\item
  Practice \texttt{pivot\_longer()} --- it's the most common
\end{itemize}

\subsection{Key takeaways}\label{key-takeaways}

\begin{enumerate}
\def\labelenumi{\arabic{enumi}.}
\tightlist
\item
  \textbf{Tidy data} has a specific structure that works with tidyverse
\item
  \textbf{pivot\_longer()} is your most-used tidying function
\item
  \textbf{pivot\_wider()} is useful for tables and some analyses
\item
  \textbf{Think about your variables} before reshaping
\item
  \textbf{Column names contain information} --- extract it with
  \texttt{names\_pattern}
\end{enumerate}

\subsection{Questions?}\label{questions}

Next time: \textbf{Data Import}

We'll learn to read CSV files, Excel spreadsheets, and more!




\end{document}
