% Options for packages loaded elsewhere
% Options for packages loaded elsewhere
\PassOptionsToPackage{unicode}{hyperref}
\PassOptionsToPackage{hyphens}{url}
\PassOptionsToPackage{dvipsnames,svgnames,x11names}{xcolor}
%
\documentclass[
  letterpaper,
  DIV=11,
  numbers=noendperiod]{scrartcl}
\usepackage{xcolor}
\usepackage{amsmath,amssymb}
\setcounter{secnumdepth}{-\maxdimen} % remove section numbering
\usepackage{iftex}
\ifPDFTeX
  \usepackage[T1]{fontenc}
  \usepackage[utf8]{inputenc}
  \usepackage{textcomp} % provide euro and other symbols
\else % if luatex or xetex
  \usepackage{unicode-math} % this also loads fontspec
  \defaultfontfeatures{Scale=MatchLowercase}
  \defaultfontfeatures[\rmfamily]{Ligatures=TeX,Scale=1}
\fi
\usepackage{lmodern}
\ifPDFTeX\else
  % xetex/luatex font selection
\fi
% Use upquote if available, for straight quotes in verbatim environments
\IfFileExists{upquote.sty}{\usepackage{upquote}}{}
\IfFileExists{microtype.sty}{% use microtype if available
  \usepackage[]{microtype}
  \UseMicrotypeSet[protrusion]{basicmath} % disable protrusion for tt fonts
}{}
\makeatletter
\@ifundefined{KOMAClassName}{% if non-KOMA class
  \IfFileExists{parskip.sty}{%
    \usepackage{parskip}
  }{% else
    \setlength{\parindent}{0pt}
    \setlength{\parskip}{6pt plus 2pt minus 1pt}}
}{% if KOMA class
  \KOMAoptions{parskip=half}}
\makeatother
% Make \paragraph and \subparagraph free-standing
\makeatletter
\ifx\paragraph\undefined\else
  \let\oldparagraph\paragraph
  \renewcommand{\paragraph}{
    \@ifstar
      \xxxParagraphStar
      \xxxParagraphNoStar
  }
  \newcommand{\xxxParagraphStar}[1]{\oldparagraph*{#1}\mbox{}}
  \newcommand{\xxxParagraphNoStar}[1]{\oldparagraph{#1}\mbox{}}
\fi
\ifx\subparagraph\undefined\else
  \let\oldsubparagraph\subparagraph
  \renewcommand{\subparagraph}{
    \@ifstar
      \xxxSubParagraphStar
      \xxxSubParagraphNoStar
  }
  \newcommand{\xxxSubParagraphStar}[1]{\oldsubparagraph*{#1}\mbox{}}
  \newcommand{\xxxSubParagraphNoStar}[1]{\oldsubparagraph{#1}\mbox{}}
\fi
\makeatother

\usepackage{color}
\usepackage{fancyvrb}
\newcommand{\VerbBar}{|}
\newcommand{\VERB}{\Verb[commandchars=\\\{\}]}
\DefineVerbatimEnvironment{Highlighting}{Verbatim}{commandchars=\\\{\}}
% Add ',fontsize=\small' for more characters per line
\usepackage{framed}
\definecolor{shadecolor}{RGB}{241,243,245}
\newenvironment{Shaded}{\begin{snugshade}}{\end{snugshade}}
\newcommand{\AlertTok}[1]{\textcolor[rgb]{0.68,0.00,0.00}{#1}}
\newcommand{\AnnotationTok}[1]{\textcolor[rgb]{0.37,0.37,0.37}{#1}}
\newcommand{\AttributeTok}[1]{\textcolor[rgb]{0.40,0.45,0.13}{#1}}
\newcommand{\BaseNTok}[1]{\textcolor[rgb]{0.68,0.00,0.00}{#1}}
\newcommand{\BuiltInTok}[1]{\textcolor[rgb]{0.00,0.23,0.31}{#1}}
\newcommand{\CharTok}[1]{\textcolor[rgb]{0.13,0.47,0.30}{#1}}
\newcommand{\CommentTok}[1]{\textcolor[rgb]{0.37,0.37,0.37}{#1}}
\newcommand{\CommentVarTok}[1]{\textcolor[rgb]{0.37,0.37,0.37}{\textit{#1}}}
\newcommand{\ConstantTok}[1]{\textcolor[rgb]{0.56,0.35,0.01}{#1}}
\newcommand{\ControlFlowTok}[1]{\textcolor[rgb]{0.00,0.23,0.31}{\textbf{#1}}}
\newcommand{\DataTypeTok}[1]{\textcolor[rgb]{0.68,0.00,0.00}{#1}}
\newcommand{\DecValTok}[1]{\textcolor[rgb]{0.68,0.00,0.00}{#1}}
\newcommand{\DocumentationTok}[1]{\textcolor[rgb]{0.37,0.37,0.37}{\textit{#1}}}
\newcommand{\ErrorTok}[1]{\textcolor[rgb]{0.68,0.00,0.00}{#1}}
\newcommand{\ExtensionTok}[1]{\textcolor[rgb]{0.00,0.23,0.31}{#1}}
\newcommand{\FloatTok}[1]{\textcolor[rgb]{0.68,0.00,0.00}{#1}}
\newcommand{\FunctionTok}[1]{\textcolor[rgb]{0.28,0.35,0.67}{#1}}
\newcommand{\ImportTok}[1]{\textcolor[rgb]{0.00,0.46,0.62}{#1}}
\newcommand{\InformationTok}[1]{\textcolor[rgb]{0.37,0.37,0.37}{#1}}
\newcommand{\KeywordTok}[1]{\textcolor[rgb]{0.00,0.23,0.31}{\textbf{#1}}}
\newcommand{\NormalTok}[1]{\textcolor[rgb]{0.00,0.23,0.31}{#1}}
\newcommand{\OperatorTok}[1]{\textcolor[rgb]{0.37,0.37,0.37}{#1}}
\newcommand{\OtherTok}[1]{\textcolor[rgb]{0.00,0.23,0.31}{#1}}
\newcommand{\PreprocessorTok}[1]{\textcolor[rgb]{0.68,0.00,0.00}{#1}}
\newcommand{\RegionMarkerTok}[1]{\textcolor[rgb]{0.00,0.23,0.31}{#1}}
\newcommand{\SpecialCharTok}[1]{\textcolor[rgb]{0.37,0.37,0.37}{#1}}
\newcommand{\SpecialStringTok}[1]{\textcolor[rgb]{0.13,0.47,0.30}{#1}}
\newcommand{\StringTok}[1]{\textcolor[rgb]{0.13,0.47,0.30}{#1}}
\newcommand{\VariableTok}[1]{\textcolor[rgb]{0.07,0.07,0.07}{#1}}
\newcommand{\VerbatimStringTok}[1]{\textcolor[rgb]{0.13,0.47,0.30}{#1}}
\newcommand{\WarningTok}[1]{\textcolor[rgb]{0.37,0.37,0.37}{\textit{#1}}}

\usepackage{longtable,booktabs,array}
\usepackage{calc} % for calculating minipage widths
% Correct order of tables after \paragraph or \subparagraph
\usepackage{etoolbox}
\makeatletter
\patchcmd\longtable{\par}{\if@noskipsec\mbox{}\fi\par}{}{}
\makeatother
% Allow footnotes in longtable head/foot
\IfFileExists{footnotehyper.sty}{\usepackage{footnotehyper}}{\usepackage{footnote}}
\makesavenoteenv{longtable}
\usepackage{graphicx}
\makeatletter
\newsavebox\pandoc@box
\newcommand*\pandocbounded[1]{% scales image to fit in text height/width
  \sbox\pandoc@box{#1}%
  \Gscale@div\@tempa{\textheight}{\dimexpr\ht\pandoc@box+\dp\pandoc@box\relax}%
  \Gscale@div\@tempb{\linewidth}{\wd\pandoc@box}%
  \ifdim\@tempb\p@<\@tempa\p@\let\@tempa\@tempb\fi% select the smaller of both
  \ifdim\@tempa\p@<\p@\scalebox{\@tempa}{\usebox\pandoc@box}%
  \else\usebox{\pandoc@box}%
  \fi%
}
% Set default figure placement to htbp
\def\fps@figure{htbp}
\makeatother





\setlength{\emergencystretch}{3em} % prevent overfull lines

\providecommand{\tightlist}{%
  \setlength{\itemsep}{0pt}\setlength{\parskip}{0pt}}



 


\KOMAoption{captions}{tableheading}
\makeatletter
\@ifpackageloaded{tcolorbox}{}{\usepackage[skins,breakable]{tcolorbox}}
\@ifpackageloaded{fontawesome5}{}{\usepackage{fontawesome5}}
\definecolor{quarto-callout-color}{HTML}{909090}
\definecolor{quarto-callout-note-color}{HTML}{0758E5}
\definecolor{quarto-callout-important-color}{HTML}{CC1914}
\definecolor{quarto-callout-warning-color}{HTML}{EB9113}
\definecolor{quarto-callout-tip-color}{HTML}{00A047}
\definecolor{quarto-callout-caution-color}{HTML}{FC5300}
\definecolor{quarto-callout-color-frame}{HTML}{acacac}
\definecolor{quarto-callout-note-color-frame}{HTML}{4582ec}
\definecolor{quarto-callout-important-color-frame}{HTML}{d9534f}
\definecolor{quarto-callout-warning-color-frame}{HTML}{f0ad4e}
\definecolor{quarto-callout-tip-color-frame}{HTML}{02b875}
\definecolor{quarto-callout-caution-color-frame}{HTML}{fd7e14}
\makeatother
\makeatletter
\@ifpackageloaded{caption}{}{\usepackage{caption}}
\AtBeginDocument{%
\ifdefined\contentsname
  \renewcommand*\contentsname{Table of contents}
\else
  \newcommand\contentsname{Table of contents}
\fi
\ifdefined\listfigurename
  \renewcommand*\listfigurename{List of Figures}
\else
  \newcommand\listfigurename{List of Figures}
\fi
\ifdefined\listtablename
  \renewcommand*\listtablename{List of Tables}
\else
  \newcommand\listtablename{List of Tables}
\fi
\ifdefined\figurename
  \renewcommand*\figurename{Figure}
\else
  \newcommand\figurename{Figure}
\fi
\ifdefined\tablename
  \renewcommand*\tablename{Table}
\else
  \newcommand\tablename{Table}
\fi
}
\@ifpackageloaded{float}{}{\usepackage{float}}
\floatstyle{ruled}
\@ifundefined{c@chapter}{\newfloat{codelisting}{h}{lop}}{\newfloat{codelisting}{h}{lop}[chapter]}
\floatname{codelisting}{Listing}
\newcommand*\listoflistings{\listof{codelisting}{List of Listings}}
\makeatother
\makeatletter
\makeatother
\makeatletter
\@ifpackageloaded{caption}{}{\usepackage{caption}}
\@ifpackageloaded{subcaption}{}{\usepackage{subcaption}}
\makeatother
\usepackage{bookmark}
\IfFileExists{xurl.sty}{\usepackage{xurl}}{} % add URL line breaks if available
\urlstyle{same}
\hypersetup{
  pdftitle={Introduction \& Setup},
  pdfauthor={Dr.~Sara Weston},
  colorlinks=true,
  linkcolor={blue},
  filecolor={Maroon},
  citecolor={Blue},
  urlcolor={Blue},
  pdfcreator={LaTeX via pandoc}}


\title{Introduction \& Setup}
\usepackage{etoolbox}
\makeatletter
\providecommand{\subtitle}[1]{% add subtitle to \maketitle
  \apptocmd{\@title}{\par {\large #1 \par}}{}{}
}
\makeatother
\subtitle{PSY 410: Data Science for Psychology}
\author{Dr.~Sara Weston}
\date{2026-03-30}
\begin{document}
\maketitle


\section{Why are we here?}\label{why-are-we-here}

\subsection{Psychology has a data
problem}\label{psychology-has-a-data-problem}

In 2015, a team of 270 researchers tried to replicate 100 published
psychology studies.

. . .

\textbf{Only 36\% produced the same results.}

. . .

This wasn't fraud. These were real labs, following published methods,
using real data.

So what went wrong?

\subsection{Many things went wrong}\label{many-things-went-wrong}

\begin{itemize}
\tightlist
\item
  Analyses that couldn't be reproduced (even by the original authors)
\item
  Data that was cleaned by hand with no record of what changed
\item
  Figures that obscured rather than revealed patterns
\item
  Code that only worked on one person's laptop
\end{itemize}

. . .

The replication crisis isn't just a statistics problem. It's a
\textbf{workflow} problem.

\subsection{This course is about the
workflow}\label{this-course-is-about-the-workflow}

Every technical skill we learn this quarter serves the same goal:

. . .

\begin{quote}
Start with raw data. End with a clear, honest, reproducible story.
\end{quote}

. . .

That means learning to:

\begin{itemize}
\tightlist
\item
  \textbf{Import} data without breaking it
\item
  \textbf{Transform} it transparently (with code, not clicks)
\item
  \textbf{Visualize} it to find patterns --- and catch errors
\item
  \textbf{Communicate} what you found so others can verify it
\end{itemize}

\subsection{About this course}\label{about-this-course}

\textbf{What we'll learn:}

\begin{itemize}
\tightlist
\item
  Data visualization with ggplot2
\item
  Data transformation with dplyr
\item
  Data tidying with tidyr
\item
  Reproducible reports with Quarto
\end{itemize}

\textbf{Why it matters:}

\begin{itemize}
\tightlist
\item
  Psychology is increasingly data-driven
\item
  These skills make your work reproducible
\item
  They transfer to any career that touches data
\end{itemize}

\subsection{The old way vs.~the new
way}\label{the-old-way-vs.-the-new-way}

\textbf{Point-and-click (SPSS, Excel)}

\begin{itemize}
\tightlist
\item
  Hard to reproduce
\item
  Error-prone
\item
  Limited visualizations
\item
  Doesn't scale
\end{itemize}

\textbf{Code-based (R)}

\begin{itemize}
\tightlist
\item
  Fully reproducible
\item
  Transparent \& shareable
\item
  Unlimited customization
\item
  Handles any data size
\end{itemize}

\subsection{Why R specifically?}\label{why-r-specifically}

\begin{longtable}[]{@{}ll@{}}
\toprule\noalign{}
What you need & R delivers \\
\midrule\noalign{}
\endhead
\bottomrule\noalign{}
\endlastfoot
Free & Open source --- no licenses, ever \\
Built for data & Created by statisticians, not software engineers \\
Psychology packages & \texttt{psych}, \texttt{lavaan}, \texttt{lme4},
\texttt{brms} \\
Publication-quality figures & \texttt{ggplot2} --- industry-leading \\
Reproducibility & Quarto integration (we'll learn this) \\
\end{longtable}

\section{Getting set up}\label{getting-set-up}

\subsection{What you need to install}\label{what-you-need-to-install}

\begin{enumerate}
\def\labelenumi{\arabic{enumi}.}
\tightlist
\item
  \textbf{R} --- the programming language

  \begin{itemize}
  \tightlist
  \item
    Download from
    \href{https://cloud.r-project.org}{cloud.r-project.org}
  \end{itemize}
\item
  \textbf{RStudio} --- the interface we'll use to write R

  \begin{itemize}
  \tightlist
  \item
    Download from
    \href{https://posit.co/download/rstudio-desktop/}{posit.co/download/rstudio-desktop}
  \end{itemize}
\end{enumerate}

. . .

\begin{tcolorbox}[enhanced jigsaw, left=2mm, rightrule=.15mm, arc=.35mm, leftrule=.75mm, colback=white, coltitle=black, colframe=quarto-callout-important-color-frame, opacityback=0, opacitybacktitle=0.6, titlerule=0mm, title=\textcolor{quarto-callout-important-color}{\faExclamation}\hspace{0.5em}{Important}, toprule=.15mm, bottomrule=.15mm, bottomtitle=1mm, breakable, toptitle=1mm, colbacktitle=quarto-callout-important-color!10!white]

Install R \textbf{first}, then RStudio. RStudio needs R to work!

\end{tcolorbox}

\subsection{R vs RStudio}\label{r-vs-rstudio}

Think of it like this:

\begin{itemize}
\tightlist
\item
  \textbf{R} is the engine of a car
\item
  \textbf{RStudio} is the dashboard, steering wheel, and GPS
\end{itemize}

You \emph{could} drive with just an engine\ldots{} but why would you?

\subsection{The RStudio interface}\label{the-rstudio-interface}

\includegraphics[width=0.8\linewidth,height=\textheight,keepaspectratio]{img/rstudio-panes.png}

The four panes: Source (top-left), Console (bottom-left), Environment
(top-right), Files/Plots/Help (bottom-right)

\subsection{The four panes}\label{the-four-panes}

\begin{longtable}[]{@{}
  >{\raggedright\arraybackslash}p{(\linewidth - 2\tabcolsep) * \real{0.3000}}
  >{\raggedright\arraybackslash}p{(\linewidth - 2\tabcolsep) * \real{0.7000}}@{}}
\toprule\noalign{}
\begin{minipage}[b]{\linewidth}\raggedright
Pane
\end{minipage} & \begin{minipage}[b]{\linewidth}\raggedright
What it does
\end{minipage} \\
\midrule\noalign{}
\endhead
\bottomrule\noalign{}
\endlastfoot
\textbf{Source} (top-left) & Write and edit your code files \\
\textbf{Console} (bottom-left) & Run code interactively, see output \\
\textbf{Environment} (top-right) & See your data and objects \\
\textbf{Files/Plots/Help} (bottom-right) & Navigate files, view plots,
get help \\
\end{longtable}

\subsection{Let's try it: The Console}\label{lets-try-it-the-console}

Type in the Console and press Enter:

\begin{Shaded}
\begin{Highlighting}[]
\DecValTok{2} \SpecialCharTok{+} \DecValTok{2}
\end{Highlighting}
\end{Shaded}

. . .

\begin{Shaded}
\begin{Highlighting}[]
\DecValTok{2} \SpecialCharTok{+} \DecValTok{2}
\end{Highlighting}
\end{Shaded}

\begin{verbatim}
[1] 4
\end{verbatim}

Congrats --- you just ran R code!

\subsection{The assignment operator}\label{the-assignment-operator}

In R, we use \texttt{\textless{}-} to assign values to objects:

\begin{Shaded}
\begin{Highlighting}[]
\NormalTok{x }\OtherTok{\textless{}{-}} \DecValTok{10}
\end{Highlighting}
\end{Shaded}

Think of it as an arrow pointing left: ``put 10 into x''

. . .

\begin{Shaded}
\begin{Highlighting}[]
\CommentTok{\# You can use objects in calculations}
\NormalTok{x }\SpecialCharTok{*} \DecValTok{2}
\NormalTok{x }\SpecialCharTok{+}\NormalTok{ x}
\end{Highlighting}
\end{Shaded}

. . .

\begin{tcolorbox}[enhanced jigsaw, left=2mm, rightrule=.15mm, arc=.35mm, leftrule=.75mm, colback=white, coltitle=black, colframe=quarto-callout-tip-color-frame, opacityback=0, opacitybacktitle=0.6, titlerule=0mm, title=\textcolor{quarto-callout-tip-color}{\faLightbulb}\hspace{0.5em}{Tip}, toprule=.15mm, bottomrule=.15mm, bottomtitle=1mm, breakable, toptitle=1mm, colbacktitle=quarto-callout-tip-color!10!white]

\textbf{Keyboard shortcut:} Alt + - (Windows) or Option + - (Mac)

\end{tcolorbox}

\section{Organizing your work}\label{organizing-your-work}

\subsection{Two mental models}\label{two-mental-models}

\subsubsection{The filing cabinet}\label{the-filing-cabinet}

Every piece of paper has a \textbf{place} where it \emph{lives}.

You navigate \textbf{to} that place to find it.

\begin{verbatim}
Documents/
├── psy410/
│   ├── data/
│   └── scripts/
└── thesis/
    ├── data/
    └── drafts/
\end{verbatim}

\subsubsection{The laundry basket}\label{the-laundry-basket}

Everything is in \textbf{one pile}.

You \textbf{search} to find it.

\emph{Google, Spotlight, Finder search, ``Recent files''\ldots{}}

. . .

Neither is wrong for daily life. But \textbf{coding requires the filing
cabinet.}

\subsection{Why coding requires the filing
cabinet}\label{why-coding-requires-the-filing-cabinet}

When you write:

\begin{Shaded}
\begin{Highlighting}[]
\NormalTok{survey }\OtherTok{\textless{}{-}} \FunctionTok{read.csv}\NormalTok{(}\StringTok{"data/raw/survey\_responses.csv"}\NormalTok{)}
\end{Highlighting}
\end{Shaded}

You're giving directions: ``Start here. Go into \texttt{data}. Then into
\texttt{raw}. Find \texttt{survey\_responses.csv}.''

. . .

If the file isn't exactly there, the code \textbf{breaks}. No fuzzy
matching. No ``did you mean\ldots?''

. . .

\begin{tcolorbox}[enhanced jigsaw, left=2mm, rightrule=.15mm, arc=.35mm, leftrule=.75mm, colback=white, coltitle=black, colframe=quarto-callout-important-color-frame, opacityback=0, opacitybacktitle=0.6, titlerule=0mm, title=\textcolor{quarto-callout-important-color}{\faExclamation}\hspace{0.5em}{Important}, toprule=.15mm, bottomrule=.15mm, bottomtitle=1mm, breakable, toptitle=1mm, colbacktitle=quarto-callout-important-color!10!white]

This is why we need to learn directory structure --- even if it feels
foreign.

\end{tcolorbox}

\subsection{Why use RStudio Projects?}\label{why-use-rstudio-projects}

\textbf{Without projects:}

\begin{itemize}
\tightlist
\item
  Files scattered everywhere
\item
  \texttt{setwd()} nightmares
\item
  ``It works on my computer''
\item
  Lost work when switching tasks
\end{itemize}

\textbf{With projects:}

\begin{itemize}
\tightlist
\item
  Everything in one folder
\item
  Paths just work
\item
  Share the whole folder
\item
  Easy to switch between projects
\end{itemize}

. . .

\begin{tcolorbox}[enhanced jigsaw, left=2mm, rightrule=.15mm, arc=.35mm, leftrule=.75mm, colback=white, coltitle=black, colframe=quarto-callout-tip-color-frame, opacityback=0, opacitybacktitle=0.6, titlerule=0mm, title=\textcolor{quarto-callout-tip-color}{\faLightbulb}\hspace{0.5em}{Tip}, toprule=.15mm, bottomrule=.15mm, bottomtitle=1mm, breakable, toptitle=1mm, colbacktitle=quarto-callout-tip-color!10!white]

\textbf{The golden rule:} Someone else should be able to run your code
and get the same results. That ``someone else'' includes Future You ---
who has forgotten everything.

\end{tcolorbox}

\subsection{Creating a project}\label{creating-a-project}

\begin{enumerate}
\def\labelenumi{\arabic{enumi}.}
\tightlist
\item
  File → New Project\ldots{}
\item
  Choose ``New Directory'' → ``New Project''
\item
  Name it (e.g., ``psy410'')
\item
  Pick a location (Documents folder is fine)
\item
  Click ``Create Project''
\end{enumerate}

. . .

You'll see a \texttt{.Rproj} file appear --- this is your project file.
From now on, double-click it to open your project.

\subsection{A standard project
structure}\label{a-standard-project-structure}

\begin{verbatim}
psy410/
├── psy410.Rproj
├── data/
│   ├── raw/           # Original data (READ ONLY)
│   └── clean/         # Processed data
├── scripts/
│   ├── 01_clean.R     # Data cleaning
│   ├── 02_analyze.R   # Main analysis
│   └── 03_visualize.R # Figures
└── output/
    └── figures/       # Saved plots
\end{verbatim}

. . .

Two principles:

\begin{enumerate}
\def\labelenumi{\arabic{enumi}.}
\tightlist
\item
  \textbf{Raw data is sacred} --- never modify original files
\item
  \textbf{Outputs are disposable} --- you can always regenerate them
  from code
\end{enumerate}

\subsection{Naming things}\label{naming-things}

Good names make code \textbf{self-documenting}. Bad names create
confusion and bugs.

\textbf{Do this:}

\begin{itemize}
\tightlist
\item
  \texttt{reaction\_time}
\item
  \texttt{mean\_anxiety}
\item
  \texttt{survey\_clean.csv}
\item
  \texttt{01\_clean\_data.R}
\end{itemize}

\textbf{Not this:}

\begin{itemize}
\tightlist
\item
  \texttt{x1}, \texttt{temp}, \texttt{foo}
\item
  \texttt{AvgAnx} (hard to read)
\item
  \texttt{data\_final\_v2\_REAL.csv}
\item
  \texttt{stuff.R}, \texttt{untitled3.R}
\end{itemize}

. . .

The test: Can someone unfamiliar with your project understand what a
variable contains or what a file does?

\subsection{R Scripts}\label{r-scripts}

R scripts (\texttt{.R} files) are where you write and save your code.

To create one:

\begin{enumerate}
\def\labelenumi{\arabic{enumi}.}
\tightlist
\item
  File → New File → R Script
\item
  Or press Ctrl/Cmd + Shift + N
\end{enumerate}

. . .

\textbf{Always save your scripts!} The console history disappears.

\subsection{Running code from a
script}\label{running-code-from-a-script}

\begin{itemize}
\tightlist
\item
  \textbf{Run one line:} Put cursor on line, press Ctrl/Cmd + Enter
\item
  \textbf{Run selection:} Highlight code, press Ctrl/Cmd + Enter
\item
  \textbf{Run entire script:} Ctrl/Cmd + Shift + Enter
\end{itemize}

. . .

\begin{tcolorbox}[enhanced jigsaw, left=2mm, rightrule=.15mm, arc=.35mm, leftrule=.75mm, colback=white, coltitle=black, colframe=quarto-callout-tip-color-frame, opacityback=0, opacitybacktitle=0.6, titlerule=0mm, title=\textcolor{quarto-callout-tip-color}{\faLightbulb}\hspace{0.5em}{Tip}, toprule=.15mm, bottomrule=.15mm, bottomtitle=1mm, breakable, toptitle=1mm, colbacktitle=quarto-callout-tip-color!10!white]

You'll use Ctrl/Cmd + Enter constantly. Memorize it!

\end{tcolorbox}

\subsection{Comments}\label{comments}

Use \texttt{\#} to write comments --- notes for yourself and others:

\begin{Shaded}
\begin{Highlighting}[]
\CommentTok{\# This calculates the mean age of participants}
\NormalTok{mean\_age }\OtherTok{\textless{}{-}} \FunctionTok{mean}\NormalTok{(ages)}

\CommentTok{\# Anything after \# is ignored by R}
\NormalTok{x }\OtherTok{\textless{}{-}} \DecValTok{10}  \CommentTok{\# this assigns 10 to x}
\end{Highlighting}
\end{Shaded}

Comments are essential. Your future self will thank you.

\section{Your first real task}\label{your-first-real-task}

\subsection{Installing packages}\label{installing-packages}

R's power comes from \textbf{packages} --- bundles of functions others
have written.

\begin{Shaded}
\begin{Highlighting}[]
\CommentTok{\# Install a package (do this once)}
\FunctionTok{install.packages}\NormalTok{(}\StringTok{"tidyverse"}\NormalTok{)}
\end{Highlighting}
\end{Shaded}

\begin{Shaded}
\begin{Highlighting}[]
\CommentTok{\# Load a package (do this every session)}
\FunctionTok{library}\NormalTok{(tidyverse)}
\end{Highlighting}
\end{Shaded}

\subsection{The tidyverse}\label{the-tidyverse}

The \texttt{tidyverse} is actually a collection of packages:

\begin{longtable}[]{@{}ll@{}}
\toprule\noalign{}
Package & Purpose \\
\midrule\noalign{}
\endhead
\bottomrule\noalign{}
\endlastfoot
\texttt{ggplot2} & Data visualization \\
\texttt{dplyr} & Data manipulation \\
\texttt{tidyr} & Data tidying \\
\texttt{readr} & Reading data files \\
\texttt{tibble} & Modern data frames \\
\texttt{stringr} & String manipulation \\
\texttt{forcats} & Working with factors \\
\texttt{purrr} & Functional programming \\
\end{longtable}

We'll use most of these throughout the course.

\subsection{Let's look at some data}\label{lets-look-at-some-data}

After loading tidyverse, you have access to built-in datasets:

\begin{Shaded}
\begin{Highlighting}[]
\FunctionTok{library}\NormalTok{(tidyverse)}
\end{Highlighting}
\end{Shaded}

\begin{verbatim}
-- Attaching core tidyverse packages ------------------------ tidyverse 2.0.0 --
v dplyr     1.1.4     v readr     2.1.5
v forcats   1.0.1     v stringr   1.5.2
v ggplot2   4.0.0     v tibble    3.3.0
v lubridate 1.9.4     v tidyr     1.3.1
v purrr     1.1.0     
-- Conflicts ------------------------------------------ tidyverse_conflicts() --
x dplyr::filter() masks stats::filter()
x dplyr::lag()    masks stats::lag()
i Use the conflicted package (<http://conflicted.r-lib.org/>) to force all conflicts to become errors
\end{verbatim}

\begin{Shaded}
\begin{Highlighting}[]
\CommentTok{\# mpg is a dataset about fuel economy}
\NormalTok{mpg}
\end{Highlighting}
\end{Shaded}

\begin{verbatim}
# A tibble: 234 x 11
   manufacturer model      displ  year   cyl trans drv     cty   hwy fl    class
   <chr>        <chr>      <dbl> <int> <int> <chr> <chr> <int> <int> <chr> <chr>
 1 audi         a4           1.8  1999     4 auto~ f        18    29 p     comp~
 2 audi         a4           1.8  1999     4 manu~ f        21    29 p     comp~
 3 audi         a4           2    2008     4 manu~ f        20    31 p     comp~
 4 audi         a4           2    2008     4 auto~ f        21    30 p     comp~
 5 audi         a4           2.8  1999     6 auto~ f        16    26 p     comp~
 6 audi         a4           2.8  1999     6 manu~ f        18    26 p     comp~
 7 audi         a4           3.1  2008     6 auto~ f        18    27 p     comp~
 8 audi         a4 quattro   1.8  1999     4 manu~ 4        18    26 p     comp~
 9 audi         a4 quattro   1.8  1999     4 auto~ 4        16    25 p     comp~
10 audi         a4 quattro   2    2008     4 manu~ 4        20    28 p     comp~
# i 224 more rows
\end{verbatim}

\subsection{Your turn!}\label{your-turn}

\begin{enumerate}
\def\labelenumi{\arabic{enumi}.}
\tightlist
\item
  Create a new RStudio Project called ``psy410''
\item
  Inside it, create three folders: \texttt{data}, \texttt{scripts},
  \texttt{output}
\item
  Create a new R script in \texttt{scripts/} and save it as
  \texttt{01\_practice.R}
\item
  Install and load the tidyverse
\item
  Run \texttt{glimpse(mpg)} and \texttt{glimpse(diamonds)}
\item
  Save your script!
\end{enumerate}

\section{Getting help}\label{getting-help}

\subsection{When you're stuck}\label{when-youre-stuck}

\begin{enumerate}
\def\labelenumi{\arabic{enumi}.}
\tightlist
\item
  \textbf{Use \texttt{?}} --- e.g., \texttt{?mean} opens the help page
\item
  \textbf{Google it} --- seriously, everyone does this
\item
  \textbf{Stack Overflow} --- most R questions are answered there
\item
  \textbf{R4DS book} --- \href{https://r4ds.hadley.nz}{r4ds.hadley.nz}
\item
  \textbf{Ask me} --- that's what I'm here for!
\end{enumerate}

\subsection{Reading help pages}\label{reading-help-pages}

\begin{Shaded}
\begin{Highlighting}[]
\NormalTok{?mean}
\end{Highlighting}
\end{Shaded}

Help pages include:

\begin{itemize}
\tightlist
\item
  \textbf{Description} --- what the function does
\item
  \textbf{Usage} --- how to call it
\item
  \textbf{Arguments} --- what inputs it takes
\item
  \textbf{Examples} --- working code you can run
\end{itemize}

The examples section is gold --- run them!

\subsection{Error messages}\label{error-messages}

Errors will happen. A lot. That's normal.

\begin{Shaded}
\begin{Highlighting}[]
\CommentTok{\# This will error {-} can you spot why?}
\FunctionTok{maen}\NormalTok{(}\FunctionTok{c}\NormalTok{(}\DecValTok{1}\NormalTok{, }\DecValTok{2}\NormalTok{, }\DecValTok{3}\NormalTok{))}
\end{Highlighting}
\end{Shaded}

\begin{verbatim}
Error in maen(c(1, 2, 3)): could not find function "maen"
\end{verbatim}

. . .

Read the error message carefully --- R is trying to help you.

\section{Why no AI? (Yet)}\label{why-no-ai-yet}

\begin{center}\rule{0.5\linewidth}{0.5pt}\end{center}

\subsubsection{The bicycle vs.~the
motorcycle}\label{the-bicycle-vs.-the-motorcycle}

Steve Jobs called the computer a \textbf{``bicycle for the mind''} ---
it amplifies your pedaling, but \emph{you} provide the balance and
direction.

Generative AI is more like a \textbf{motorcycle}. It provides the
engine. But if you don't know how to ride, you crash faster and harder.

Read Cat Hicks:
\href{https://www.fightforthehuman.com/cognitive-helmets-for-the-ai-bicycle-part-1/}{``Cognitive
helmets for AI bicycles''}

\begin{center}\rule{0.5\linewidth}{0.5pt}\end{center}

\subsubsection{The metacognition trap}\label{the-metacognition-trap}

AI gives you a working answer immediately. It \emph{feels} like you
solved it.

. . .

But you didn't build the \textbf{mental model} of \emph{why} it works.

. . .

When the AI hallucinates (which it will), \textbf{you cannot debug it.}
You are stranded.

Using AI now = skipping the gym but expecting to get strong.

\begin{center}\rule{0.5\linewidth}{0.5pt}\end{center}

\subsubsection{Building the helmet
first}\label{building-the-helmet-first}

Before we get on the motorcycle, we need a \textbf{cognitive helmet}:

\begin{enumerate}
\def\labelenumi{\arabic{enumi}.}
\tightlist
\item
  \textbf{Syntax literacy} --- reading code as fluently as English
\item
  \textbf{Debugging resilience} --- emotional regulation when things
  break
\item
  \textbf{Strategic friction} --- breaking big problems into small
  pieces
\end{enumerate}

. . .

That's what the many short assignments in this course build. The goal:
\textbf{you} are the pilot, not the passenger.

\begin{center}\rule{0.5\linewidth}{0.5pt}\end{center}

\section{Wrapping up}\label{wrapping-up}

\subsection{Before next class}\label{before-next-class}

\textbf{Read:}

\begin{itemize}
\tightlist
\item
  \href{https://r4ds.hadley.nz/workflow-basics}{R4DS Ch 2: Workflow
  basics}
\item
  \href{https://r4ds.hadley.nz/workflow-scripts}{R4DS Ch 6: Workflow:
  scripts and projects}
\end{itemize}

\textbf{Do:}

\begin{itemize}
\tightlist
\item
  Install R and RStudio
\item
  Create your psy410 project with \texttt{data/}, \texttt{scripts/}, and
  \texttt{output/} folders
\item
  Install the tidyverse package
\item
  Poke around! Try things! Break stuff!
\end{itemize}

\subsection{Key takeaways}\label{key-takeaways}

\begin{enumerate}
\def\labelenumi{\arabic{enumi}.}
\tightlist
\item
  \textbf{Psychology needs better data practices} --- that's what this
  course builds
\item
  \textbf{R is a tool} --- like learning any tool, it takes practice
\item
  \textbf{Organization matters} --- filing cabinet, not laundry basket
\item
  \textbf{Name things clearly} --- your future self will thank you
\item
  \textbf{Errors are learning opportunities} --- read them, Google them
\end{enumerate}

\subsection{The one thing to remember}\label{the-one-thing-to-remember}

Every skill you learn in this course is a brick in a wall between you
and the replication crisis.

See you Wednesday for your first visualization!




\end{document}
